\chapter{Smiths algorithm}
\label{ch:algorithm}

% Short introduction to the chapter (max 1/2 page)
The following chapter will describe the different parts of the algorithm used by
Smith~\cite{Smith1992}.

\section{Overview}
\label{sec:overview}

The algorithm in general follow the form of as proposed by Gilbert and
Pollak~\missingref{18 in Smith}

\TODO[inline]{Finish the overview}

\TODO[inline]{Add a note that while there may not have been other ways to go at
  the time, today we also have things such as the one described in Fonseca,
  Pawel, et al.}

\section{Topologies}
\label{sec:topologies}

The first step of the algorithm is, as described in \Cref{sec:overview}, to
generate topologies. It is therefore natural that we need some way of
representing and generating these topologies.

The algorithm only considers full Steiner topologies where $K = N - 2$. This is
a \NOTE{is there a better word than legal?} legal approach as we can simply
regard any Steiner tree with $K \le N - 2$ as a \ac{FST} where some edges have
length zero and thus some points have ``merged''.

\subsection{Representation}
\label{sec:representation}

\begin{figure}[htbp]
\centering
  \begin{subfigure}[b]{0.3\textwidth}
    \includegraphics[width=\textwidth]{gfx/tikz/algorithm-topology-1}
    \caption{Here be dragons.\label{fig:algorithm-topology-1}}
  \end{subfigure}
~
  \begin{subfigure}[b]{0.3\textwidth}
    \includegraphics[width=\textwidth]{gfx/tikz/algorithm-topology-2}
    \caption{Here be dragons.\label{fig:algorithm-topology-2}}
  \end{subfigure}
~
  \begin{subfigure}[b]{0.3\textwidth}
    \includegraphics[width=\textwidth]{gfx/tikz/algorithm-topology-3}
    \caption{Here be dragons.\label{fig:algorithm-topology-3}}
  \end{subfigure}
\caption[How to construct Steiner topologies]{Here be dragons.\label{fig:algorithm-topologies}}
\end{figure}

There is an 1--1 correspondence between full Steiner topologies with $N \ge 3$
regular points, and $(N-3)$-vectors $\vec{a}$, whose $i$th entry $a_i$ is an
integer between $1 \le a_i \le 2 i + 1$.

\TODO[inline]{Describe/Outline proof and the generation as shown in the figures.}

\subsection{Generation}
\label{sec:generation}

Using the representation described in \Cref{sec:representation} the problem of
generating all topologies can now be done as a backtracking problem generating
all $(N-3)$-topology vectors.

\TODO[inline]{Describe the backtracking here}

To further speed up the generation of topologies, or rather to avoid generating
unnecessary topologies, Smith also utilizes the following theorem

\begin{theorem}
For any set of $N$ distinct regular points in any Euclidean space, the length of
the shortest tree, interconnecting $N-1$ points, with topology vector $a_1
\cdots a_{N-4}$ is no greater than the length of the shortest tree,
interconnecting $N$ points, with topology vector $a_1 \cdots a_{N-3}$.
\end{theorem}

\TODO[inline]{Proof of above theorem}

The algorithm utilizes this to prune in the following way. Imagine we have found
some upper bound for the \ac{SMT}. If we then optimize any generated topology
vector which does not yet include all the regular points, and it turns out to
have length greater than the upper bound, we can prune any topologies that we
would have generated from this vector, as the length of the larger topologies
cannot become any smaller than the length of the current, and thus cannot become
smaller than the length of the upper bound.

Thus the implementation of the algorithm generates and optimizes topologies
depth-first to ensure we get a upper bound as quickly as possible. If it did
breadth-first we would not be able to prune anything, as we would get all the
full topologies as the last to optimize.

\section{Optimization of a topology}
\label{sec:optim-topol}

\TODO[inline]{TODO}

\subsection{Iteration}
\label{sec:iter-fixm-could}

\FIXME[inline]{This section could probably be better named}
\TODO[inline]{TODO}

%%% Local Variables:
%%% mode: latex
%%% TeX-master: "../main"
%%% End:
