{ \abnormalparskip{0pt}
\chapter{Proof of the Analytical Method for the Fermat-Torricelli Problem}
\label{cha:proof-analyt-meth}
}

\section{2D}
\label{sec:2d-1}

\begin{proof}
To prove \cref{eq:15} the first step is to prove the two following
equations
%
\begin{align}
  \label{eq:16}
  K_1 K_2 + K_1 K_3 + K_2 K_3 &= 2 \sqrt{3} S d\\
  \label{eq:2}
  r_{23}^2 K_1 + r_{13}^2 K_2 + r_{12}^2 K_3 &= 2 S d
\end{align}
%
Both can proven by simply substituting the right hand sides with the definition
of each symbol, i.e.\ through direct computation. Using \cref{eq:16} we can
rewrite \cref{eq:15} as
%
\begin{gather}
  \label{eq:17}
  \left\{
    \begin{array}{c}
  x_\ast = \frac{1}{\frac{1}{K_1} + \frac{1}{K_2} + \frac{1}{K_3}} \left( \frac{x_1}{K_1} +
    \frac{x_2}{K_2} + \frac{x_3}{K_3} \right) \\
  y_\ast = \frac{1}{\frac{1}{K_1} + \frac{1}{K_2} + \frac{1}{K_3}} \left( \frac{y_1}{K_1} +
    \frac{y_2}{K_2} + \frac{y_3}{K_3} \right)
    \end{array}
  \right.  \intertext{Which will be useful in later calculations. The second
    part is to prove that}
  \sqrt{{(x_\ast - x_j)}^2 + {(y_\ast - y_j)}^2} = \frac{K_j}{\sqrt{3 d}},
\quad j \in \{ 1, 2, 3 \}\label{eq:18}
\end{gather}
%
As the calculations are similar for all $j$, only the one for $j = 1$ is shown
here. Thus
%
\begin{align}
  (x_\ast &- x_1)^2
  + {(y_\ast - y_1)}^2 \\
  &\stackrel{\mathclap{(\ref{eq:15})}}{=}
    {\left( \frac{K_1 K_2 K_3}{2 \sqrt{3} S d} \left( \frac{x_1}{K_1} +
    \frac{x_2}{K_2} + \frac{x_3}{K_3} \right) - x_1 \right)}^2 +
    {\left( \frac{K_1 K_2 K_3}{2 \sqrt{3} S d} \left( \frac{y_1}{K_1} +
    \frac{y_2}{K_2} + \frac{y_3}{K_3} \right) - y_1 \right)}^2 \\
  &= {\left( \frac{K_1 K_2 K_3}{2 \sqrt{3} S d} \right)}^2 \left[{\left(
    \frac{x_1}{K_1} + \frac{x_2}{K_2} + \frac{x_3}{K_3} -
    \frac{x_1 2 \sqrt{3} S d}{K_1 K_2 K_3} \right)}^2 + {\left(
    \frac{y_1}{K_1} + \frac{y_2}{K_2} + \frac{y_3}{K_3} -
    \frac{y_1 2 \sqrt{3} S d}{K_1 K_2 K_3} \right)}^2 \right] \\
  &\stackrel{\mathclap{(\ref{eq:16})}}{=}
    {\left( \frac{K_1 K_2 K_3}{2 \sqrt{3} S d} \right)}^2
    \left[ {\left( \frac{x_2}{K_2} + \frac{x_3}{K_3} - \frac{x_1}{K_2} -
    \frac{x_1}{K_3} \right)}^2 + {\left( \frac{y_2}{K_2} + \frac{y_3}{K_3} -
    \frac{y_1}{K_2} - \frac{y_1}{K_3} \right)}^2 \right] \\
  &= {\left( \frac{K_1 K_2 K_3}{2 \sqrt{3} S d} \right)}^2
    \left[ \frac{{(x_2 - x_1)}^2 + {(y_2 - y_1)}^2}{K_2^2} +
    \frac{{(x_3 - x_1)}^2 + {(y_3 - y_1)}^2}{K_3^2} \right. \\
  &\hspace{7em} \left. + \; 2 \frac{(x_2 - x_1)(x_3 - x_1) +
    (y_2 - y_1)(y_3 - y_1)}{K_2 K_3} \right] \\
  &\stackrel{\mathclap{(\ref{eq:1})}}{=}
    {\left( \frac{K_1 K_2 K_3}{2 \sqrt{3} S d} \right)}^2
    \left[ \frac{r_{12}^2}{K_2^2} + \frac{r_{13}^2}{K_3^2} +
    \frac{r_{12}^2 + r_{13}^2 - r_{23}^2}{K_2 K_3} \right] \label{eq:30} \\
  &= \frac{K_1^2}{{(2 \sqrt{3} S d)}^2}
    \left[ r_{12}^2 K_3^2 + r_{13}^2 K_2^2 +
    (r_{12}^2 + r_{13}^2 - r_{23}^2) K_2 K_3 \right] \\
  &= \frac{K_1^2}{{(2 \sqrt{3} S d)}^2}
    \left[ (r_{12}^2 K_3 + r_{13}^2 K_2) (K_2 + K_3) -
    r_{23}^2 K_2 K_3 \right] \\
  &\stackrel{\mathclap{(\ref{eq:2})}}{=}
    \frac{K_1^2}{{(2 \sqrt{3} S d)}^2}
    \left[ (2 S d - r_{23}^2 K_1) (K_2 + K_3) - r_{23}^2 K_2 K_3 \right] \\
  &= \frac{K_1^2}{{(2 \sqrt{3} S d)}^2}
    \left[ 2 S d (K_2 + K_3) - r_{23}^2 (K_1 K_2 + K_1 K_3 + K_2 K_3) \right] \\
  &\stackrel{\mathclap{(\ref{eq:16})}}{=}
    \frac{K_1^2}{{(2 \sqrt{3} S d)}^2}
    \left[ 2 S d (K_2 + K_3) - 2 \sqrt{3} r_{23}^2 S d \right] \\
  &= \frac{2 S d K_1^2}{{(2 \sqrt{3} S d)}^2}
    \left[ K_2 + K_3 - \sqrt{3} r_{23}^2 \right] \\
  &\stackrel{\mathclap{(\ref{eq:3})}}{=}
    \frac{K_1^2}{6 S d}
    \left[ 2 S \right] \\
  &= \frac{K_1^2}{3 d}
\end{align}
%
Thus
%
\begin{equation}
  \label{eq:29}
  \sqrt{{(x_\ast - x_1)}^2 + {(y_\ast - y_1)}^2}
  = \sqrt{\frac{K_1^2}{3 d}} = \frac{K_1}{\sqrt{3 d}}
\end{equation}
%
To finish the proof one would also have to show that $K_1, K_2, K_3$ are
nonnegative. The proof of this can be found in~\cite[p.~5-6]{uteshev2014}.

We can now prove \cref{eq:15}, by substituting it into \cref{eq:13}. As the
equations are similar for $x$ and $y$, only $x$ is shown here
%
\begin{align}
  0
  &= \frac{(x_\ast - x_1)}{\sqrt{{(x_\ast - x_1)}^2 + {(y_\ast - y_1)}^2}} +
    \frac{(x_\ast - x_2)}{\sqrt{{(x_\ast - x_2)}^2 + {(y_\ast - y_2)}^2}} +
    \frac{(x_\ast - x_3)}{\sqrt{{(x_\ast - x_3)}^2 + {(y_\ast - y_3)}^2}} \\
  &\stackrel{\mathclap{(\ref{eq:18})}}=
    \frac{(x_\ast - x_1)}{\frac{K_1}{\sqrt{3 d}}} +
    \frac{(x_\ast - x_2)}{\frac{K_2}{\sqrt{3 d}}} +
    \frac{(x_\ast - x_3)}{\frac{K_3}{\sqrt{3 d}}} \\
  &= \sqrt{3 d} \left[
    \frac{x_\ast - x_1}{K_1} +
    \frac{x_\ast - x_2}{K_2} +
    \frac{x_\ast - x_3}{K_3} \right] \\
  &= \sqrt{3 d} \left[
    x_\ast \left( \frac{1}{K_1} + \frac{1}{K_2} + \frac{1}{K_3} \right) -
    \left( \frac{x_1}{K_1} + \frac{x_2}{K_2} + \frac{x_3}{K_3} \right)
    \right] \\
  &\stackrel{\mathclap{(\ref{eq:17})}}=
    \sqrt{3 d} \left[
    \frac{1}{\frac{1}{K_1} + \frac{1}{K_2} + \frac{1}{K_3}} \left( \frac{x_1}{K_1} +
    \frac{x_2}{K_2} + \frac{x_3}{K_3} \right)
    \left( \frac{1}{K_1} + \frac{1}{K_2} + \frac{1}{K_3} \right) -
    \left( \frac{x_1}{K_1} + \frac{x_2}{K_2} + \frac{x_3}{K_3} \right)
    \right] \\
  &= \sqrt{3 d} \left[
    \frac{1}{\frac{1}{K_1} + \frac{1}{K_2} + \frac{1}{K_3}}
    \left( \frac{1}{K_1} + \frac{1}{K_2} + \frac{1}{K_3} \right) -
    1 \right] \\
  &= \sqrt{3 d} \left[ 1 - 1 \right] = 0
\end{align}
%
As can be seen \cref{eq:15} is a solution to \cref{eq:13}, and thus it minimizes
$F(x,y)$.
\end{proof}

\section{$d$-Space}
\label{sec:d-space}

\begin{proof}
As before to prove \cref{eq:5} (and thus \cref{eq:8}) the first step once
again is to prove \cref{eq:16} and \cref{eq:2}. These can as before be
established through direct computation, i.e.\ by simply substituting the
expressions for their definitions, until both sides only contain distances,
after which the two sides can be seen to be equal. Using \cref{eq:16} we can
rewrite \cref{eq:5} as
%
\begin{gather}
    x_{(\ast, j)} = \frac{1}{\frac{1}{K_1} + \frac{1}{K_2} + \frac{1}{K_3}} \left( \frac{x_{(1,j)}}{K_1} +
  \frac{x_{(2,j)}}{K_2} + \frac{x_{(3,j)}}{K_3} \right), \quad 0 < j \le
d\label{eq:10}
\intertext{similarly to \cref{eq:17} in the 2D version. Secondly we wish
  to prove that}
  \sqrt{\sum_{i = 1}^d {(x_{(\ast,i)} - x_{j,i})}^2} = \frac{K_j}{\sqrt{3 d}
  }, \quad j \in \{ 1, 2, 3 \}\label{eq:19}
\end{gather}
As the calculations are similar for all $j$, only the one for $j = 1$ is shown
here. Thus
%
\begin{align}
  \sum_{i=1}^d (x_{(\ast,i)} &- x_{(1,i)})^2 \\
  &= \sum_{i=1}^d { \left( \frac{K_1 K_2 K_3}{2 \sqrt{3} S d}
    \left( \frac{x_{(1,i)}}{K_1} + \frac{x_{(2,i)}}{K_2} + \frac{x_{(3,i)}}{K_3}
    \right) - x_{(1,i)} \right) }^2 \\
  &= { \left( \frac{K_1 K_2 K_3}{2 \sqrt{3} S d} \right) }^2 \left[ \sum_{i=1}^d
    { \left( \frac{x_{(1,i)}}{K_1} + \frac{x_{(2,i)}}{K_2} +
    \frac{x_{(3,i)}}{K_3} - \frac{x_{(1,i)} 2 \sqrt{3} S d}{K_1 K_2 K_3}
    \right) }^2 \right] \\
  &\stackrel{\mathclap{(\ref{eq:16})}}{=}
    {\left( \frac{K_1 K_2 K_3}{2 \sqrt{3} S d} \right)}^2 \left[ \sum_{i=1}^d
    {\left( \frac{x_{(2,i)}}{K_2} + \frac{x_{(3,i)}}{K_3} - \frac{x_{(1,i)}}{K_2} -
    \frac{x_{(1,i)}}{K_3} \right)}^2 \right] \\
  &= {\left( \frac{K_1 K_2 K_3}{2 \sqrt{3} S d} \right)}^2 \left[ \sum_{i=1}^d
    \left( \frac{{(x_{(2,i)} - x_{(1,i)})}^2}{K_2^2} +
    \frac{{(x_{(3,i)} - x_{(1,i)})}^2}{K_3^2} \right. \right. \\
  &\hspace{7em} \left. \left. + \; 2 \frac{(x_{(2,i)} - x_{(1,i)})(x_{(3,i)} -
    x_{(1,i)})}{K_2 K_3} \right) \right] \\
  &= {\left( \frac{K_1 K_2 K_3}{2 \sqrt{3} S d} \right)}^2
    \left[ \frac{\sum_{i=1}^d {(x_{(2,i)} - x_{(1,i)})}^2}{K_2^2} +
    \frac{\sum_{i=1}^d {(x_{(3,i)} - x_{(1,i)})}^2}{K_3^2} \right. \\
  &+ \left. \frac{ \sum_{i=1}^d {(x_{(2,i)} - x_{(1,i)})}^2 +
    \sum_{i=1}^d {(x_{(3,i)} - x_{(1,i)})}^2 -
    \sum_{i=0}^d {(x_{(2,i)} - x_{(3,i)})}^2}{K_2 K_3} \right] \\
  &\stackrel{\mathclap{(\ref{eq:14})}}{=}
    {\left( \frac{K_1 K_2 K_3}{2 \sqrt{3} S d} \right)}^2
    \left[ \frac{r_{12}^2}{K_2^2} + \frac{r_{13}^2}{K_3^2} +
    \frac{r_{12}^2 + r_{13}^2 - r_{23}^2}{K_2 K_3} \right] \\
  \intertext{As this point we have exactly the same as in \cref{eq:30} and thus
  we can conclude}
  &= \frac{K_1^2}{3 d}
\end{align}
%
From here we can show that \cref{eq:5} is a solution to \cref{eq:7} in the
exactly the same way as
we proved that \cref{eq:15} was a solution to \cref{eq:13}, by substituting
\cref{eq:5} into \cref{eq:7} and using \cref{eq:10} and \cref{eq:19}.
\end{proof}

%%% Local Variables:
%%% mode: latex
%%% TeX-master: "../../main"
%%% End:
