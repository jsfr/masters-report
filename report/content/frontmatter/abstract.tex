\begin{abstract}
This thesis presents possible improvements to the original implementation of the
algorithm for finding \aclp{smt} proposed by \textcite{smith1992}. The suggested changes include a new data structure and method
for building topologies as well as a method for ordering the terminals before the
topologies. This is used to avoid double-work in the original
implementation which rebuilds the entire tree every time its topology changes.

This the thesis also presents a simple iteration for
optimizing \aclp{fst} as an alternative to the one proposed by
\textcite{smith1992}. This includes presenting an analytical solution to the
Fermat-Torricelli problem as proposed by \textcite{uteshev2014} and further
generalizing it from 2D to higher dimensions.

% Apart from presenting the preliminaries for understanding the \acl{estp} and the
% algorithm proposed in \textcite{smith1992}, the thesis also presents and
% discusses elements of \textcite{smith1992} that are questionable or odd. This
% includes both elements of the article and the actual implementation given in the
% article.

The thesis also presents experimental work performed to test the correctness and
efficiency (compared to the original implementation) of the new implementation.
The new suggested iteration, based on the analytical solution, shows some
sub-optimal results. These are discussed and further investigated, and are
concluded to probably stem from the used error-function and the if-clauses
deciding when to prune a topology. A possible way of fixing the sub-optimality
is then proposed.

With the sub-optimality in mind, the new simple iteration however seems to show
promise as the performance of it in general is much better than the original
implementation using \citeauthor{smith1992}'s iteration. The new implementation
of \citeauthor{smith1992}'s iteration show performance equivalent to
the original implementation. This however is ascribed to the choice of programming
language, and the much larger amount of trees the new implementation optimizes
before reaching optimality. The new method for ordering terminals shows very promising results
(when used with either of the iterations).

Finally the thesis presents and discusses possible extensions and improvements
to the new implementation.
\end{abstract}

%%% Local Variables:
%%% mode: latex
%%% TeX-master: "../../main"
%%% End:
