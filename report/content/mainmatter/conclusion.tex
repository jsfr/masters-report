{
\abnormalparskip{0pt}
\chapter{Conclusion}
\label{cha:conclusion}
}

The thesis has presented the \ac{estp} and preliminaries for understanding it.
The thesis has also presented methods for finding \acp{smt} in two dimensions
and have described why these do not generalize to higher dimensions. Known methods for
finding \acp{smt} in higher dimensions have been presented and apart from the
branch algorithm by \textcite{fonseca2014} all of these have been seen to be
modifications or extension to the method proposed by \textcite{smith1992}.

The thesis has presented the algorithm for finding \acp{smt} in euclidean
$d$-space proposed by \textcite{smith1992}. Apart from presenting it, the thesis
has also identified and discussed elements of the article and implementation
given in~\cite{smith1992} which were questionable. The most distinctive of which
were the choice of error function for which no real argument has been given and
the if-clauses when pruning which were seen to possibly cause sub-optimality in
the new implementation.

The thesis has presented improvements to the \citeauthor{smith1992}'s algorithm
and implementation. The first were to use a new method and data structure for
building topologies to avoid rebuilding the tree every time the topology is
extended. The second were a new method for ordering the terminals before before
building the topologies. This method greedily selected the terminals furthest
apart. Finally the thesis have presented a new simple iteration as an
alternative to the iteration given by \textcite{smith1992} for optimizing the
tree of a full Steiner topology. The presentation of this simple method included
the presentation and generalization to higher dimensions of the analytical
solution to the Fermat-Torricelli problem for $3$ points presented in 2D by
\textcite{uteshev2014}. This iteration was proposed as \textcite{smith1992} had
noted one such method would converge slowly but gave no data or proof to back up
the claim. As the method was computationally very light it seemed likely that it
would perform similarly or better than the iteration proposed by
\textcite{smith1992}.

Furthermore the thesis has accounted for the structure of
the new implementation which has been structured such that most of the work lies
in a self-contained library for others to continue to work on or with.

The thesis has presented the experimental work performed to test the efficiency
and correctness of the new implementation (i.e.\ both of the iterations, with
and without sorting of terminals). The correctness tests revealed that
sub-optimality would occur in some instances when using the new simple
iteration. This sub-optimality was discussed and further investigated.
Eventually the sub-optimality was ascribed mostly to the use of current
error-function on the entire tree, and the if-clauses used when optimizing and
pruning topologies. The thesis afterwards presented the tests done to test the
efficiency of the new implementation when compared with the original
implementation. The tests in general showed promising results for the new simple
iteration which on most occasions was much faster than the original
implementation. The new implementation of \citeauthor{smith1992}'s iteration in
general performed slightly worse than the original. The cause for this is still
unknown, but several qualified suggestions were made. These were the choice of
programming language and more importantly a bug/change which causes the new
implementation to optimize on many more trees than the original implementation.
This higher number of optimizations was ascribed to the fact that the new
implementation reuses the coordinates from the previous optimized tree, causing
it to move very slowly. The new method for ordering terminals also showed good
performance. It brought the speed of the simple iteration even further
below the original implementation and the speed of the new implementation of
\citeauthor{smith1992}'s iteration at the same level/slightly below the original
implementation.

Furthermore the performance tests seemed to debunk the claim by
\textcite{smith1992} that a simple iteration would converge much slower than the
one he proposes.

Finally the thesis has a discussion of possible extensions and improvements for
the new implementation. The first was the possibility of making the
implementation concurrent, for which a possible implementation scheme was
outlined. The second was more efficient way of propagating changes out through
the tree when optimizing with the simple iteration. It was described how this
could speed up convergence by allowing us to only optimize on the part of the
tree which is changes. The third and final improvement was to exchange the
initial placement of Steiner points in the implementation with the perturbed
Fermat-Torricelli point, instead of the perturbed centroid. A test which showed
much improved numbers for iterations and optimized trees was performed to
further support the argument of making this change.

%%% Local Variables:
%%% mode: latex
%%% TeX-master: "../../main"
%%% End:
