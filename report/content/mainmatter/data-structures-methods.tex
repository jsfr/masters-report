 {
\abnormalparskip{0pt}
\chapter{Data Structures and Methods}
\label{cha:data-structures}
}

This chapter will describe some more notable elements of the new implementation,
with regard to data structures and methods not used by
\textcite{smith1992}. Specifically it describes the new data structure
introduced for holding and building topologies. It also describes the new
iteration based on the simple iteration described in \cref{sec:simple-iteration}
and using the analytical method for solving the Fermat-Torricelli problem
presented by \textcite{uteshev2014}. This also includes a description of said
method, and a generalization to $d$-space, as the one presented by
\textcite{uteshev2014} only is for 2D. Finally the chapter also describes a new
way of sorting the terminals.

\section{Building Topologies}
\label{sec:building-topologies}

In \citeauthor{smith1992}'s implementation of the algorithm when generating
topologies every new topology is rebuilt completely from the bottom. E.g.\ say
that we have just optimized the topology corresponding to the topology vector
$A = [1, 2]$. We now wish to optimize the topology vector
$A' = [1, 2, 3]$. In the old implementation $A'$ would have to be built again
from scratch instead of simply splitting and building on to $A$.

To avoid this extra work, the new implementation instead uses a different
approach. The underlying data of a tree is in most aspects more or less
identical to \citeauthor{smith1992}'s as it is a series of arrays to hold points
and edges. However, these are coupled together in a struct (in Go, the
programming language of choice, this is essentially a record on which one can add
methods.). Instead of rebuilding this entire struct every time we split or
restore the topology, it has methods to do exactly this. Thus, using the earlier
example, from $A$ we could either split an
edge and get $A'$, or any other length three topology vector having $A$ as a
prefix, or we could restore and go back from $A$ to $A'' = [1]$\footnote{From
  $A''$ we could then either split again to e.g.\ the next topology vector of
  length two, $[1, 3]$, or restore once again.}

The way this is implemented is a follows: All edges of the tree are kept in n
list. An edge is represented as the tuple $(a, b)$ meaning it starts at
point $p_a$ and ends at point $p_b$. Say we have $n$ terminals in all, and have
currently connected the $i$ first terminals. This means we have $i-2$ Steiner points
continuously enumerated from $n+1$ up to $n+i-1$ and $2i-3$ edges enumerated
continuously from $1$.

Now we wish to split some edge $e_j = (c, d)$ where $1 \le j \le 2i-3$, so any
of the existing edges, and connect the next terminal $p_{i+1}$ to the existing
topology. We do this by first changing the original edge to go from the first
end point $p_c$ to a new Steiner point $p_{n+i}$. Afterwards we append two new
edges to the list of edges, $e_{2i-2} = (d, n+i)$ and $e_{2i-1} = (i+1, n+i)$
thus connecting $p_d$ and $p_{i+1}$ to Steiner point $p_{n+i}$ as well. This
process is straight forward and if the operation of accessing and appending to
the list is done in $\mathcal{O}(1)$ time\footnote{which is the case if enough
  space is allocated for the list ahead of the appending.} then the entire
operation will be done in $\mathcal{O}(1)$ time. The operation can be seen in
\cref{fig:splitting-topology}. In this way we can expand the topology one
terminal at a time without rebuilding every time.

\begin{figure}[htbp]
  \centering
  \begin{tabular}{|c|ccc|c|ccc|c|}
  \multicolumn{1}{c}{$1$}
  & \multicolumn{3}{c}{ }
  & \multicolumn{1}{c}{$i$}
  & \multicolumn{3}{c}{ }
  & \multicolumn{1}{c}{$2k + 1$}
  \\ \hhline{--~---~--}
  $(a,b)$
  & & \ldots
  & & \cellcolor{SkyBlue} \tikzmark{a} $(c,d)$
  & & \ldots
  & & $(e,f)$
  \\ \hhline{--~---~--}
\end{tabular}

\vspace{8ex}

\begin{tabular}{|c|ccc|c|ccc|c|c|c|}
  \multicolumn{1}{c}{$1$}
  & \multicolumn{3}{c}{ }
  & \multicolumn{1}{c}{$i$}
  & \multicolumn{3}{c}{ }
  & \multicolumn{1}{c}{$2k+1$}
  & \multicolumn{1}{c}{$2k+2$}
  & \multicolumn{1}{c}{$2k+3$}
  \\ \hhline{--~---~----}
  $(a,b)$
  & & \ldots
  & & \cellcolor{SkyBlue} \tikzmark{b} $(c,k+1)$
  & & \ldots
  & & $(e,f)$
  & \cellcolor{SkyBlue} \tikzmark{c} $(d,k+1)$
    & \cellcolor{SkyBlue} \tikzmark{d} $(g,k+1)$
  \\ \hhline{--~---~----}
\end{tabular}

\begin{tikzpicture}[overlay, remember picture, yshift=-1.5ex, xshift=2em]
  \draw [-{Latex[length=2mm,width=2mm]},semithick] ({pic cs:a}) [bend right] to ([yshift=6.5ex]{pic cs:b});
  \draw [-{Latex[length=2mm,width=2mm]},semithick] ({pic cs:a}) [out=-70, in=110] to ([yshift=6.5ex]{pic cs:c});
  \draw [-{Latex[length=2mm,width=2mm]},semithick] ({pic cs:a}) [out=-20, in= 120] to ([yshift=6.5ex]{pic cs:d});
\end{tikzpicture}

%%% Local Variables:
%%% mode: latex
%%% TeX-master: "../main"
%%% End:

  \caption[Implementation of topology splitting]{Splitting on edge $e_j$ with
    $i$ terminals currently connected, results in two new edges $e_{2i-2}$ and
    $e_{2i-1}$ and a change of the existing edge $e_j$. Finally we also get the new
    Steiner point $n+i$.\label{fig:splitting-topology}}
\end{figure}

The reason for this way of expanding the topology however lies with the wish to
also restore an edge. If we keep a list of all the edges we split
on\footnote{which we do, as this is the topology vector described in
  \cref{sec:representation}.} we can always restore the last edge split in the
following way. Select the last two edges, $(d, n+i), (i+1, n+i)$ in the list and
the edge $e_j = (c, n+i)$. Now change edge $e_j$ to $(c, d)$ and remove the last
two edges in the list. Thus we have $\mathcal{O}(1)$-time methods for splitting and
restoring the topology, in a way which is consistent in the numbering with the
one described in \cref{sec:representation} and shown in
\cref{fig:algorithm-topologies}.

\section{Analytical Solution to the Fermat-Torricelli Problem}
\label{sec:analyt-solut-ferm}

An analytical solution for finding the Fermat-Torricelli point of three
prespecified points in 2D is presented by \textcite{uteshev2014}---which is the same as
finding the Steiner point of those three points. The article presents a
solution for two dimensions which will be described shortly. However as we are working
in $\mathbb{R}^d$ a generalization of the solution to $d$-space is necessary and
will also be presented. This generalization is as far as I know, not presented
anywhere else in the literature.

The reason why this is interesting is that \textcite{smith1992} mentions that an
iterative process which updates points one at a time would not be as quick to
converge as the one he presents. There is however no experiments in the paper to
support this. Thus this analytical method is interesting, as it is actually not
computationally heavy to compute the location of a Steiner point, and as each
optimization moves a Steiner point to the exact location at which it minimizes
the length to its three neighbors at the time. One could therefore think that this might
actually converge quite fast.

Using this analytical solution an optimization iteration similar to the one
described in \cref{sec:simple-iteration} has been implemented. The method
roughly follows the one described, but also re-uses the main-loop of the method
using \citeauthor{smith1992}'s iteration, with its conditions for pruning and
stopping.

\subsection{2D}
\label{sec:2d}

The solution for two dimensions is presented by \textcite{uteshev2014} is for
the generalized Fermat-Torricelli problem, which given three non-colinear points
$p_1 = (x_1, y_1)$, $p_2 = (x_2, y_2)$ and $p_3 = (x_3, y_3)$ in the plane,
finds the point $p_\ast = (x_\ast, y_\ast)$ which gives a solution to the
optimization problem
%
\begin{equation}
  \min_{(x,y)} F(x,y) \quad \text{for} \quad F(x,y) = \sum_{j=1}^3 m_j
  \sqrt{{(x-x_j)}^2 + {(y-y_j)}^2}
\end{equation}
%
Where the weight of the $j$th point, $m_j$, is a real positive number. The
instance where $m_1 = m_2 = m_3 = 1$ is exactly the classical Fermat-Torricelli
problem, which has a unique solution either coinciding with one of the points
$p_1$, $p_2$, $p_3$ or with the Fermat-Torricelli or Steiner point of the
triangle $p_1 p_2 p_3$. The instance with unequal weights is known as the
generalized Fermat-Torricelli point.

As we only need the classical version of the problem, I have decided to rewrite
the proof done by \textcite{uteshev2014} for the 2D case with the weights set to
$1$. In a lot of ways the proof is the same, though it does become a bit
simpler and cleaner. The original proof for the generalized version can be found
in \textcite{uteshev2014}. Thus what we wish to minimize is instead the
following problem:
%
\begin{equation}
  \min_{(x,y)} F(x,y) \quad \text{for} \quad F(x,y) = \sum_{j=1}^3
  \sqrt{{(x-x_j)}^2 + {(y-y_j)}^2}
\end{equation}
%
Existence and uniqueness of the solution is guaranteed by
\cref{thm:fermat-torricelli}
%
\begin{theorem}
\label{thm:fermat-torricelli}
  Denote the corner angles of the triangle $\triangle p_1 p_2 p_3$ by $\alpha_1, \alpha_2,
  \alpha_3$. Then if the conditions
  %
  \begin{equation}
    \label{eq:12}
    \left\{
      \begin{array}{c}
        1 < 2 + 2 \cos \alpha_1 , \\
        1 < 2 + 2 \cos \alpha_2 , \\
        1 < 2 + 2 \cos \alpha_3 , \\
      \end{array}
    \right.
  \end{equation}
  %
  are fulfilled, corresponding to all angles being less than $120^{\circ}$, then
  there exists a unique solution $p_\ast = (x_\ast, y_\ast) \in \mathbb{R}^2$
  for the classic Fermat-Torricelli problem lying inside the triangle
  $\triangle p_1 p_2 p_3$. This point is a stationary point for the function
  $F(x,y)$, i.e.\ a real solution of the system
  %
  \begin{equation}
    \label{eq:13}
    \left\{
      \begin{array}{c}
        \frac{d(F(x,y))}{dx} = \sum_{j=1}^3 \frac{(x-x_j)}{\sqrt{{(x-x_j)}^2 +
        {(y - y_j)}^2}} = 0, \\
        \frac{d(F(x,y))}{dy} = \sum_{j=1}^3 \frac{(y-y_j)}{\sqrt{{(x-x_j)}^2 +
        {(y - y_j)}^2}} = 0
      \end{array}
    \right.
  \end{equation}
  %
  If any of the conditions in \cref{eq:12} is violated then $F(x,y)$ attains its
  minimum value at the corresponding vertex of triangle.
\end{theorem}
%
Calculating the coordinates of the point $(x_\ast, y_\ast)$, if the conditions
in \cref{eq:12} are fulfilled, is done as follows
\begin{gather}
  \label{eq:15}
  x_\ast = \frac{K_1 K_2 K_3}{2 \sqrt{3} S d} \left( \frac{x_1}{K_1} +
  \frac{x_2}{K_2} + \frac{x_3}{K_3} \right), \quad
  y_\ast = \frac{K_1 K_2 K_3}{2 \sqrt{3} S d} \left( \frac{y_1}{K_1} +
  \frac{y_2}{K_2} + \frac{y_3}{K_3} \right)
  \intertext{with}
  \label{eq:11}
  F(x_\ast, y_\ast) = \min_{(x,y)} F(x,y) = \sqrt{d}
  \intertext{Here}
  \label{eq:1}
  r_{jl} = \sqrt{{(x_j - x_l)}^2 + {(y_j - y_l)}^2} = |p_j p_l|
  \quad \{j,l\} \in \{ 1, 2, 3 \} \\
  \label{eq:20}
  S = | x_1 y_2 + x_2 y_3 + x_3 y_1 - x_1 y_3 -x_3 y_2 - x_2 y_1 | \\
  \label{eq:3}
  \left\{
    \begin{array}{c}
      K_1 = \frac{\sqrt{3}}{2} (r_{12}^2 + r_{13}^2 - r_{23}^2) + S
      \\
      K_2 = \frac{\sqrt{3}}{2} (r_{12}^2 + r_{23}^2 - r_{13}^2) + S
      \\
      K_3 = \frac{\sqrt{3}}{2} (r_{23}^2 + r_{13}^2 - r_{12}^2) + S
    \end{array}
  \right. \\
  d = \frac{1}{\sqrt{3}} (K_1 + K_2 + K_3) \\
\end{gather}

The proof of \cref{eq:15,eq:11} can be found in a generalized version in
\textcite{uteshev2014}, or in \cref{sec:2d-1} where it has been
adapted to the unweighted version of the problem.

\subsection{Generalization to $d$-Space}
\label{sec:gener-high-dimens}

Generalizing the solution described above to higher dimensions is in general
pretty straightforward. The biggest change is in the representation of $S$, and
the fact that the proof requires a bit of sum manipulation. This generalized
version of the proof is not known to have been presented in any paper before.

The generalization is as for the 2D variant here done for the classic
Fermat-Torricelli problem. It could however just as easily be done for the
weighted problem, by following the form of the proof in \textcite{uteshev2014}
instead of the form in \cref{sec:2d}.

The first change is that we no longer try to minimize a function of just two
parameters, but instead we intend to minimize a function of $d$ parameters,
where our points are contained in $\mathbb{R}^d$, i.e.\ we have $d$ dimensions. As we
normally represent this using vectors, we would not actually have $d$
parameters, but instead one parameter being a $d$-vector. Thus we wish to find a
point $p_\ast = (x_{(\ast,1)}, x_{(\ast,2)}, \ldots, x_{(\ast,d)}) \in \mathbb{R}^d$
which minimizes the following function for the three points
$p_1 = (x_{(1,1)}, x_{(1,2)}, \ldots, x_{(1,d)}), p_2 = (x_{(2,1)}, x_{(2,2)},
\ldots, x_{(2,d)}), p_3 = (x_{(3,1)}, x_{(3,2)}, \ldots, x_{(3,d)}) \in \mathbb{R}^d$.
%
\begin{gather}
  \min_{(x_1, x_2, \ldots, x_d)} F(x_1, x_2, \ldots, x_d) \quad \text{for} \quad
  F(x_1, x_2, \ldots, x_d) = \sum_{j=1}^3 \sqrt{\sum_{i=1}^d {(x_i -
    x_{(j,i)})}^2 }
  \\ \Updownarrow \\
  \label{eq:4}
  \min_{(p)} F(p) \quad \text{for} \quad
  F(p) = \sum_{j=1}^3 | p p_j |, \quad p \in \mathbb{R}^d
\end{gather}
%
The point $p_\ast$ now no longer has to be a solution to the equation system in
\cref{eq:13}, but to a similar system containing $d$ equations, as follows
%
\begin{equation}
\label{eq:7}
  \left.
    \begin{array}{c}
    \sum_{j=1}^3 \frac{(x_1-x_{(j,1)})}{| p p_j|} = 0
    \\
    \sum_{j=1}^3 \frac{(x_2-x_{(j,2)})}{| p p_j|} = 0
    \\
    \vdots
    \\
    \sum_{j=1}^3 \frac{(x_d-x_{(j,d)})}{| p p_j|} = 0
  \end{array}
  \right\} =
  \sum_{j=1}^3 \frac{(p - p_j)}{| p p_j |} = \vec{0}
\end{equation}
%
Then as before, if the conditions in \cref{eq:12} are fulfilled\footnote{If not
  the point is as before located at the corresponding vertex.} the solution to
\cref{eq:7} can be found as follows
%
\begin{gather}
  p_\ast = (x_{(\ast,1)}, x_{(\ast,2)}, \ldots, x_{(\ast,d)})\label{eq:8}
  \intertext{with}
  x_{(\ast, j)} = \frac{K_1 K_2 K_3}{2 \sqrt{3} S d} \left( \frac{x_{(1,j)}}{K_1} +
  \frac{x_{(2,j)}}{K_2} + \frac{x_{(3,j)}}{K_3} \right), \quad 0 < j \le d\label{eq:5}
  \intertext{and}
  F(p_\ast) = \min_{(p)} F(p) = \sqrt{d}\label{eq:21}
  \intertext{Here}
  \label{eq:14} r_{jl} = \sqrt{\sum_{i = 1}^d {(x_{(j,i)} - x_{(l,i)})}^2} = |p_j p_l|
  \quad \{j,l\} \in \{ 1, 2, 3 \} \\
  S = \frac{1}{2} \sqrt{(r_{12} + r_{23} + r_{13}) (r_{23} + r_{13}) (r_{12} +
    r_{13}) (r_{12} + r_{23})} \\
\end{gather}
%
And the rest of the expressions are the same as in the 2D version. As can be
seen, all of the expressions, except for $S$ (and the distance which now of
course have more than two dimensions), retain the same representation as in the
two dimensional version. $S$ however looks quite different. Exactly how the new
$S$ is related to the old $S$ is still unclear to me. \textcite{uteshev2014}
describes that in 2D if one rewrites $S$ it can be recognized as the doubled
area of the triangle $\triangle p_1 p_2 p_3$. This geometrical understanding
however seems to be missing from the higher dimension case. Indeed the idea for
the representation of $S$ stems from talks with my supervisor Pawel Winter, who
pointed me to an earlier implementation of the general case, that he has done
himself. Thus an intuition of the new $S$ is missing, however as can be seen
from the proof \cref{sec:d-space}, it does indeed work.

\section{Sorting the Terminals}
\label{sec:sorting-terminals}

The idea of sorting the terminals before one starts to build and optimize
topologies is not new, and has also been suggested by \textcite{smith1992}. The
idea is to find some way of sorting the terminals, such that the initial
\acp{rmt} becomes as large as possible, as fast as possible. Doing this
allows us to early on see if we need to discard a topology vector as prefix, if
its \ac{rmt} is longer than the upper bound, and thus be able to prune more
topologies than if we added terminals in an arbitrary order.

At least two different kinds of sorting has been implemented by
\textcite{fonseca2014,vanlaarhoven2013}. Both methods have shown promising
results regarding the number of trees one needs to optimize in contrast to not
sorting the terminals.

The first method, presented by \textcite{vanlaarhoven2013}, sorts the terminals
by their decreasing distance to the centroid of the set of terminals, i.e.\ the
first terminal is the farthest away, and the last terminal is the closest. A
potential issue with this way of sorting, is that while the terminals may be far
from the centroid, they can potentially still be rather close to each other,
thus not pruning much faster.

The second method, presented by \textcite{fonseca2014}, sorts the terminals such
that the first three terminals $p_1, p_2, p_3$ maximize the sum of their
pairwise distances, and $p_i, i = 4, 5, \ldots, n$ is the farthest away from
$p_1, p_2, \ldots, p_{i-1}$. This method seems to have a lot of potential as it
tries to ensure that every terminal added is as far away from the current
\ac{rmt} as possible. However one could easily think of positions for the
terminals, such that the terminals $p_4, p_5, \ldots, p_n$ all lie close to each
other, and within the first three terminals. This could mean that the distance
added every step after the initial \ac{rmt} will be spread out on the next few
steps, instead of adding as much as possible as quickly as possible.

I propose another method that is relatively similar to the one proposed and
implemented by \textcite{fonseca2014}, but a lot simpler and requiring fewer
steps. The initial idea for the method was to find the longest path in a fully
connected graph, where the nodes in the graph are the terminals. This problem is
however NP-hard~\cite{longestpathproblem}, and it seems a bit silly to solve a
NP-hard problem to try and improve on a NP-hard problem. Thus the method
proposed here sorts the terminals as follows: Select the initial terminal $p_1$
arbitrarily. From there select the next terminal $p_i, i = 2, 3, \ldots, n$ such
that the distance to $p_{i-1}$ is maximized.

The idea behind this approach, is that the method is extremely easy to implement
and hopefully gives better runtimes, comparable to those of e.g.\
\textcite{fonseca2014}.

%%% Local Variables:
%%% mode: latex
%%% TeX-master: "../../main"
%%% End:
