{
\abnormalparskip{0pt}
\chapter{Experiments}
\label{cha:experiments}
}

Introduction, max 1/2--1 page

golang 1.4.1/2

Naming of methods:
\begin{table}[htbp]
  \centering
  \begin{tabular}{cp{9cm}}
    \toprule
    Method       & Description                                      \\
    \midrule
    Simple       & New implementation with the analytical method.   \\
    SimpleSort   & New implementation with the analytical method and sorted
                   terminals.                                       \\
    SmithNew     & New implementation with \citeauthor{smith1992}'s
                   iteration.                                       \\
    SmithNewSort & New implementation with \citeauthor{smith1992}'s iteration
                   and sorted terminals.                            \\
    SmithOld     & Original implementation by \textcite{smith1992}. \\
    \bottomrule
  \end{tabular}
  \caption[Here be dragons]{Here be dragons.\label{tab:method-names}}
\end{table}

\section{Correctness}
\label{sec:correctness}

$150$ random cubes with $n = 10 \ldots 12$ and $d = 2$.

Simple seems faulty in a few instances:

\begin{table}[htbp]
  \centering
  \begin{tabular}{ccc}
    \toprule
    Instance           & Method     & Diff        \\
    \midrule
    cube\_n12\_d2\_s27 & Simple     & $0.0032114$ \\
    cube\_n12\_d2\_s49 & Simple     & $0.0062502$ \\
    cube\_n10\_d2\_s42 & SimpleSort & $0.0002782$ \\
    cube\_n12\_d2\_s26 & SimpleSort & $0.0040131$ \\
    cube\_n12\_d2\_s43 & SimpleSort & $0.0065104$ \\
    cube\_n12\_d2\_s49 & SimpleSort & $0.0062502$ \\
    \bottomrule
  \end{tabular}
  \caption[Here be dragons]{Here be dragons.\label{tab:correctness-errors}}
\end{table}

This seems to be because the current error function is incompatible with this
iteration, as the error gets very low quickly, but may have some of the
described problems. Thus the line \texttt{q-r < upperBound} will skip some trees
which can actually become better. The same goes for the lines deciding when we
stop optimization.

\section{Speed}
\label{sec:speed}

\begin{table}[htbp]
  \centering
  \begin{tabular}{ccccc}
    \toprule
    Set     & Dimensions       & Terminals             & Set size & Point configuration \\
    \midrule
    Carioca & $d = 3 \ldots 5$ & $n = 11 \ldots 16$    & $90$       & Random in cube      \\
    Cube    & $d = 2 \ldots 4$ & $n = 10 \ldots 15$    & $360$      & Random in cube      \\
    Iowa05  & $d = 3 \ldots 5$ & $n = 10$              & $30$       & Random in cube      \\
    Sausage & $d = 2 \ldots 5$ & $n = 10 \ldots 15$    & $24$       & Simplex sequence    \\
    Solids  & $d = 3$          & $n = 4, 6, 8, 12, 20$ & $5$        & Platonic solids     \\
    \bottomrule
  \end{tabular}
  \caption[Here be dragons]{Here be dragons.\label{tab:test-sets}}
\end{table}

\begin{table}[htbp]
  \centering
  \begin{tabular}{cccccc}
    \toprule
            & \multicolumn{5}{c}{Method}                               \\
    \cmidrule(l){2-6}
    Set     & Simple & SimpleSort & SmithNew & SmithNewSort & SmithOld \\
    \cmidrule(r){1-1}\cmidrule(l){2-6}
    Carioca & $81$   & $89$       & $73$     & $87$         & $76$     \\
    Cube    & $360$  & $360$      & $352$    & $360$        & $356$    \\
    Iowa05  & $30$   & $30$       & $30$     & $30$         & $30$     \\
    Sausage & $22$   & $23$       & $18$     & $22$         & $21$     \\
    Solids  & $4$    & $4$        & $4$      & $4$          & $4$      \\
    \bottomrule
  \end{tabular}
  \caption[Here be dragons]{Here be dragons.\label{tab:set-success}}
\end{table}

All sets have been taken from \textcite{fonseca2014}, but the sets Carioca and
Cube has been pruned.

8 $\times$ Intel\textsuperscript{\textregistered}
Xeon\textsuperscript{\textregistered} CPU E5-2630L @ 2.00 GHz and 16 GB RAM

Describe why I have pruned Carioca and Cube.

Plot all five against each other on interesting sets

\begin{figure}[htbp]
  \centering
  \begin{subfigure}[t]{0.5\textwidth}
    \includegraphics[width=\textwidth]{gfx/boxplots/plot_nvst_boxplot_d3_Carioca_1}
  \caption{Here be dragons.\label{fig:boxplot-d3-carioca-1}}
  \end{subfigure}%
  \begin{subfigure}[t]{0.5\textwidth}
    \includegraphics[width=\textwidth]{gfx/boxplots/plot_nvst_boxplot_d4_Carioca_1}
  \caption{Here be dragons.\label{fig:boxplot-d3-carioca-1}}
  \end{subfigure}%
  \caption[Here be dragons]{Here be dragons\label{fig:boxplot-carioca-1}}
\end{figure}

\begin{figure}[htbp]
  \centering
  \begin{subfigure}[t]{0.5\textwidth}
    \includegraphics[width=\textwidth]{gfx/boxplots/plot_nvst_boxplot_d3_Cube_1}
  \caption{Here be dragons.\label{fig:boxplot-d3-cube-1}}
  \end{subfigure}%
  \begin{subfigure}[t]{0.5\textwidth}
    \includegraphics[width=\textwidth]{gfx/boxplots/plot_nvst_boxplot_d4_Cube_1}
  \caption{Here be dragons.\label{fig:boxplot-d3-cube-1}}
  \end{subfigure}%
  \caption[Here be dragons]{Here be dragons\label{fig:boxplot-cube-1}}
\end{figure}

\begin{figure}[htbp]
  \centering
  \begin{subfigure}[t]{0.5\textwidth}
    \includegraphics[width=\textwidth]{gfx/boxplots/plot_nvst_boxplot_d3_Cube_2}
  \caption{Here be dragons.\label{fig:boxplot-d3-cube-2}}
  \end{subfigure}%
  \begin{subfigure}[t]{0.5\textwidth}
    \includegraphics[width=\textwidth]{gfx/boxplots/plot_nvst_boxplot_d4_Cube_2}
  \caption{Here be dragons.\label{fig:boxplot-d3-cube-2}}
  \end{subfigure}%
  \caption[Here be dragons]{Here be dragons\label{fig:boxplot-cube-2}}
\end{figure}

\begin{figure}[htbp]
  \centering
  \begin{subfigure}[t]{0.5\textwidth}
    \includegraphics[width=\textwidth]{gfx/boxplots/plot_nvst_boxplot_d3_Cube_3}
  \caption{Here be dragons.\label{fig:boxplot-d3-cube-3}}
  \end{subfigure}%
  \begin{subfigure}[t]{0.5\textwidth}
    \includegraphics[width=\textwidth]{gfx/boxplots/plot_nvst_boxplot_d4_Cube_3}
  \caption{Here be dragons.\label{fig:boxplot-d3-cube-3}}
  \end{subfigure}%
  \caption[Here be dragons]{Here be dragons\label{fig:boxplot-cube-3}}
\end{figure}

\section{Trees and Iterations}
\label{sec:trees-iterations}

Introduction

\subsection{Trees}
\label{sec:trees}

\begin{table}[htbp]
  \centering
  \begin{tabular}{ccccccc}
    \toprule
         &     & \multicolumn{5}{c}{Method}                               \\
    \cmidrule(l){3-7}
    $n$  & $d$ & Simple & SimpleSort & SmithNew & SmithNewSort & SmithOld \\
    \cmidrule(r){1-2}\cmidrule(l){3-7}
    $10$ & $2$ & $1.04$ & $0.13$     & $1.42$   & $0.17$       & $1.00$   \\
         & $3$ & $1.21$ & $0.25$     & $1.70$   & $0.37$       & $1.00$   \\
         & $4$ & $1.17$ & $0.60$     & $1.61$   & $0.89$       & $1.00$   \\
         & $5$ & $1.54$ & $0.85$     & $2.25$   & $1.34$       & $1.00$   \\
    \cmidrule(r){1-2}
    $11$ & $2$ & $1.05$ & $0.10$     & $1.62$   & $0.13$       & $1.00$   \\
         & $3$ & $1.23$ & $0.19$     & $1.90$   & $0.29$       & $1.00$   \\
         & $4$ & $1.47$ & $0.36$     & $1.97$   & $0.56$       & $1.00$   \\
         & $5$ & $1.54$ & $0.77$     & $2.57$   & $1.22$       & $1.00$   \\
    \cmidrule(r){1-2}
    $12$ & $2$ & $1.17$ & $0.08$     & $1.90$   & $0.11$       & $1.00$   \\
         & $3$ & $1.42$ & $0.16$     & $2.34$   & $0.25$       & $1.00$   \\
         & $4$ & $1.73$ & $0.26$     & $2.83$   & $0.52$       & $1.00$   \\
         & $5$ & $1.53$ & $0.59$     & $2.24$   & $1.11$       & $1.00$   \\
    \cmidrule(r){1-2}
    $13$ & $2$ & $1.11$ & $0.05$     & $2.22$   & $0.09$       & $1.00$   \\
         & $3$ & $1.43$ & $0.12$     & $2.80$   & $0.20$       & $1.00$   \\
         & $4$ & $0.74$ & $0.25$     & $3.19$   & $0.43$       & $1.00$   \\
         & $5$ & $1.66$ & $0.45$     &          & $1.14$       & $1.00$   \\
    \cmidrule(r){1-2}
    $14$ & $2$ & $0.97$ & $0.05$     & $2.63$   & $0.08$       & $1.00$   \\
         & $3$ & $1.61$ & $0.10$     & $3.44$   & $0.19$       & $1.00$   \\
         & $4$ & $1.52$ & $0.16$     &          & $0.39$       & $1.00$   \\
         & $5$ &        &            &          &              &          \\
    \cmidrule(r){1-2}
    $15$ & $2$ & $1.32$ & $0.04$     & $3.12$   & $0.06$       & $1.00$   \\
         & $3$ &        &            &          &              &          \\
         & $4$ &        &            &          &              &          \\
         & $5$ &        &            &          &              &          \\
    \bottomrule
  \end{tabular}
  \caption[Here be dragons]{Here be dragons.\label{tab:trees-sausage}}
\end{table}

\subsection{Iterations}
\label{sec:iterations}

\begin{table}[htbp]
  \centering
  \small
  \begin{tabular}{cccccc}
    \toprule
         & \multicolumn{5}{c}{Method}                               \\
    \cmidrule(l){2-6}
    $n$  & Simple & SimpleSort & SmithNew & SmithNewSort & SmithOld \\
    \cmidrule(r){1-1}\cmidrule(l){2-6}
    $4$  & $0.24$ & $0.25$     & $0.99$   & $0.99$       & $1.00$   \\
    $6$  & $0.34$ & $0.33$     & $0.98$   & $1.01$       & $1.00$   \\
    $8$  & $0.28$ & $0.26$     & $1.38$   & $1.19$       & $1.00$   \\
    $12$ & $0.17$ & $0.12$     & $1.49$   & $1.51$       & $1.00$   \\
    $20$ &        &            &          &              &          \\
    \bottomrule
  \end{tabular}
  \caption[Here be dragons]{Here be dragons.\label{tab:iterations-solids-ratio}}
\end{table}

%%% Local Variables:
%%% mode: latex
%%% TeX-master: "../../main"
%%% End:
