{
\abnormalparskip{0pt}
\chapter{Experiments}
\label{cha:experiments}
}

Introduction, max 1/2--1 page

\section{Correctness}
\label{sec:correctness}

$150$ random cubes with $n = 10 \ldots 12$ and $d = 2$.

Simple seems faulty in a few instances:
- cube\_n12\_d2\_s26.txt: Here it has a Steiner point lying atop another which in
smiths has been moved out from it a bit. Thus it seems the new iteration has
some situations with the error which it does not catch but is caugth by smiths
perturbations

- cube\_n10\_d2\_s42

- cube\_n12\_d2\_s49

\section{Speed}
\label{sec:speed}

Naming of methods:
\begin{itemize}
\item \textbf{Simple} \quad New implementation with the analytical method:
  \texttt{steinertree~-iteration=simple}
\item \textbf{SimpleSort} \quad New implementation with the analytical method
  and sorted terminals: \texttt{steinertree~-iteration=simple~-sort}
\item \textbf{SmithNew} \quad New implementation with \citeauthor{smith1992}'s
  iteration: \texttt{steinertree~-iteration=smith}
\item \textbf{SmithNewSort} \quad New implementation with \citeauthor{smith1992}'s iteration and
  sorted terminals: \texttt{steinertree~-iteration=smith~-sort}
  \item \textbf{SmithOld} \quad Original implementation by \textcite{smith1992}.
\end{itemize}

\begin{table}[htbp]
  \centering
  \begin{tabular}{ccccc}
    \toprule
    Set name & Dimensions       & Terminals             & Set size & Point configuration \\
    \midrule
    Carioca  & $d = 3 \ldots 5$ & $n = 11 \ldots 16$    & 90       & Random in cube      \\
    Cube     & $d = 2 \ldots 4$ & $n = 10 \ldots 15$    & 360      & Random in cube      \\
    Iowa05   & $d = 3 \ldots 5$ & $n = 10$              & 30       & Random in cube      \\
    Sausage  & $d = 2 \ldots 5$ & $n = 10 \ldots 15$    & 24       & Simplex sequence    \\
    Solids   & $d = 3$          & $n = 4, 6, 8, 12, 20$ & 5        & Platonic solids     \\
    \bottomrule
  \end{tabular}
  \caption[Here be dragons]{Here be dragons.\label{tab:test-sets}}
\end{table}

\begin{table}[htbp]
  \centering
  \begin{tabularx}{1.0\linewidth}{cccccc}
    \toprule
            & Simple & SimpleSort & SmithNew & SmithNewSort & SmithOld \\
    \midrule
    Carioca & $81$   & $89$       & $73$     & $87$         & $76$     \\
    Cube    & $360$  & $360$      & $352$    & $360$        & $356$    \\
    Iowa05  & $30$   & $30$       & $30$     & $30$         & $30$     \\
    Sausage & $22$   & $23$       & $18$     & $22$         & $21$     \\
    Solids  & $4$    & $4$        & $4$      & $4$          & $4$      \\
    \bottomrule
  \end{tabularx}
  \caption[Here be dragons]{Here be dragons.\label{tab:set-success}}
\end{table}

All sets have been taken from \textcite{fonseca2014}, but the sets Carioca and
Cube has been pruned.

8 $\times$ Intel\textsuperscript{\textregistered}
Xeon\textsuperscript{\textregistered} CPU E5-2630L @ 2.00 GHz and 16 GB RAM

Describe why I have pruned Carioca and Cube.

Plot all five against each other on interesting sets

\section{Trees and Iterations}
\label{sec:trees-iterations}

Introduction

\subsection{Trees}
\label{sec:trees}

SmithNew vs SmithNewSorted vs SmithOld

Simple vs SimpleSort

\subsection{Iterations}
\label{sec:iterations}

SmithNew vs Simple

Maybe write that the iterations from SmithOld is unusable as some are negative
due to overflow.

%%% Local Variables:
%%% mode: latex
%%% TeX-master: "../../main"
%%% End:
