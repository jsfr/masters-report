 {
\abnormalparskip{0pt}
\chapter{Implementation}
\label{cha:implementation}
}

The new implementation has been written in the, still relatively new but at this
point stable, language Go, which quoting the homepage: ``Go is an open source
programming language that makes it easy to build simple, reliable, and efficient
software.''~\cite{golanghomepage}. Go is a general-purpose, strongly typed and
garbage-collected language. The syntax bears resemblance to C\footnote{But
  with a lot less braces, and no pointer arithmetic.}. Furthermore Go has
explicit support for concurrent programming. The language furthermore sports
extremely fast compiling of source code and tries to have performance
comparable to C.

The reasoning behind choosing Go for the new implementation were firstly that I
already had experience with the language making it easy to get to work on the
new implementation without also having to learn a new programming
language. Secondly the language has, in my opinion, a very clean syntax which
makes it easily readable for anyone familiar with any C-syntax styled
language. Finally the easy use of concurrency in Go\footnote{Starting a new
  goroutine, which is the languages type of lightweight threads, is as simple as
  writing: \texttt{go~SomeFunction~(args\ldots)}.} made it interesting as
continued work with the new implementation could include a potential,
significant, speedup by making the branching of the algorithm concurrent. The
same could be be true for the Gauss elimination, concurrent\footnote{Not
  concurrency has actually been implemented in the new implementation. However
  the possibilities of this is discussed in \cref{sec:concurrency}.}.

In general the implementation tries to follow guidelines given in
\textcite{effectivego}. These are guidelines as to how one codes effective
Go. Here effective means both in terms of speed, memory usage, stability and
readability.

Source code for the new implementation, fixes for the old implementation and
experiments can be found at \url{https://github.com/jsfr/SteinerExact}. The
source code for just the new implementation is also included in
\cref{cha:source-code}.

This chapter will describe the architecture of the new implementation, i.e.\ how
the new implementation has been structured.

\TODO[inline]{Describe the source code architecture, and rewrite introduction so
it makes sense with new structure.}

\section{Main and Config File}
\label{sec:main-file}

\section{\acs{smt} Package}
\label{sec:smt-package}

\subsection{Data Structures}
\label{sec:data-structures}

\subsection{Methods}
\label{sec:methods}

%%% Local Variables:
%%% mode: latex
%%% TeX-master: "../../main"
%%% End:
