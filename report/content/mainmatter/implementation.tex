 {
\abnormalparskip{0pt}
\chapter{Implementation}
\label{cha:implementation}
}

\TODO[inline]{Intro text}

The new implementation has been written in the, still relatively new but at this
point stable, language Go, which quoting the homepage: ``Go is an open source
programming language that makes it easy to build simple, reliable, and efficient
software.''~\cite{golanghomepage}. Go is a general-purpose, strongly typed and
garbage-collected language. The syntax bears resemblance to to C\footnote{But
  with a lot less braces, and no pointer arithmetic.}. Furthermore Go has
explicit support for concurrent programming. The language furthermore sports
extremely fast compiling of source code and in tries to have performance
comparable to C.

The reasoning behind choosing Go for the new implementation were the following:
Firstly I already had experience with the language making it easy to get to work
on the new implementation without also having to learn a new programming
language. Secondly the language has, in my opinion, a very clean syntax which
makes it easily readable for anyone familiar with any C-syntax styled
language. Thirdly the

Finally the easy use of concurrency in Go\footnote{Starting a new goroutine is
  as simple as writing: \texttt{go~SomeFunction~(args\ldots)}.} made it
interesting as potential speedup could be achieved by making the branching
of the program concurrent.

In general the implementation tries to follow guidelines given in
\textcite{effectivego}.

The new implementation, fixes for the old implementation and experiments can be
found at \url{https://github.com/jsfr/SteinerExact}. The source
code for just the new implementation is also included in \cref{cha:source-code}.

\section{Overview}
\label{sec:overview-1}

\subsection{Architecture}
\label{sec:architecture}

\section{Building topologies}
\label{sec:building-topologies}

In \textcite{smith1992} implementation of the algorithm when generating
topologies every new topology is rebuilt completely from the bottom. E.g.\ say
that we have just optimized the topology corresponding to the topology vector
$A_1 = (1, 2)$. If we now wish to optimize the topology vector
$A_2 = (1, 2, 3)$. In the old implementation $A_2$ would have to be built again
from scratch instead of simply splitting and building on to $A_1$. To try and do
this is in a slightly more clever way the new implementation is instead able to
either split or restore a topology, thus from $A_1$ we could either split an
edge and get $A_2$, or any other length three topology vector having $A_1$ as a
prefix, or we could restore and go back from $A_1$ to $A_0 = (1)$\footnote{From
  $A_0$ we could then either split again to e.g.\ the next length two topology
  vector $A_1' = (1, 3)$ or restore once again.}

The way this is implemented is a follows: All edges of the tree is kept in an
list, and edge being represented as the tuple $(a, b)$ meaning it starts at
point $p_a$ and ends at point $p_b$. Say we have $n$ terminals in all, and have
currently connected the $i$ first terminals. This means we $i-2$ Steiner points
continuously enumerated from $n+1$ up to $n+i-1$ and $2i-3$ edges enumerated
continuously from $1$.

Now we wish to split some edge $e_j = (c, d)$ where $1 \le j \le 2i-3$, so any
of the existing edges, and connect the next terminal $p_{i+1}$ to the existing
topology. We do this by first changing the original edge to go from the first
end point $p_c$ to a new Steiner point $p_{n+i}$. Afterwards we append two new
edges to the list of edges, $e_{2i-2} = (d, n+i)$ and $e_{2i-1} = (i+1, n+i)$
thus connecting $p_d$ and $p_{i+1}$ to Steiner point $p_{n+i}$ as well. This
process is straight forward and if the accessing into the list and
appending\footnote{which is both the case if enough space is allocated for the
  list ahead of the appending.} is $\mathcal{O}(1)$ then the entire operation
will be $\mathcal{O}(1)$. The operation can be seen in
\cref{fig:splitting-topology}. In this way we can expand the topology one
terminal at a time without rebuilding every time.

\begin{figure}[htbp]
  \centering
  \begin{tabular}{|c|ccc|c|ccc|c|}
  \multicolumn{1}{c}{$1$}
  & \multicolumn{3}{c}{ }
  & \multicolumn{1}{c}{$i$}
  & \multicolumn{3}{c}{ }
  & \multicolumn{1}{c}{$2k + 1$}
  \\ \hhline{--~---~--}
  $(a,b)$
  & & \ldots
  & & \cellcolor{SkyBlue} \tikzmark{a} $(c,d)$
  & & \ldots
  & & $(e,f)$
  \\ \hhline{--~---~--}
\end{tabular}

\vspace{8ex}

\begin{tabular}{|c|ccc|c|ccc|c|c|c|}
  \multicolumn{1}{c}{$1$}
  & \multicolumn{3}{c}{ }
  & \multicolumn{1}{c}{$i$}
  & \multicolumn{3}{c}{ }
  & \multicolumn{1}{c}{$2k+1$}
  & \multicolumn{1}{c}{$2k+2$}
  & \multicolumn{1}{c}{$2k+3$}
  \\ \hhline{--~---~----}
  $(a,b)$
  & & \ldots
  & & \cellcolor{SkyBlue} \tikzmark{b} $(c,k+1)$
  & & \ldots
  & & $(e,f)$
  & \cellcolor{SkyBlue} \tikzmark{c} $(d,k+1)$
    & \cellcolor{SkyBlue} \tikzmark{d} $(g,k+1)$
  \\ \hhline{--~---~----}
\end{tabular}

\begin{tikzpicture}[overlay, remember picture, yshift=-1.5ex, xshift=2em]
  \draw [-{Latex[length=2mm,width=2mm]},semithick] ({pic cs:a}) [bend right] to ([yshift=6.5ex]{pic cs:b});
  \draw [-{Latex[length=2mm,width=2mm]},semithick] ({pic cs:a}) [out=-70, in=110] to ([yshift=6.5ex]{pic cs:c});
  \draw [-{Latex[length=2mm,width=2mm]},semithick] ({pic cs:a}) [out=-20, in= 120] to ([yshift=6.5ex]{pic cs:d});
\end{tikzpicture}

%%% Local Variables:
%%% mode: latex
%%% TeX-master: "../main"
%%% End:

  \caption[Implementation of topology splitting]{Splitting on edge $e_j$ with
    $i$ terminals currently connected, results in two new edges $e_{2i-2}$ and
    $e_{2i-1}$ and a change of the existing edge $e_j$. Finally we also get the new
    Steiner point $n+i$.\label{fig:splitting-topology}}
\end{figure}

The reason for this way of expanding the topology however lies with the wish to
also restore an edge. If we keep a list of all the edges we split
on\footnote{which we do, as this is the topology vector described in
  \cref{sec:representation}.} we can always restore the last edge split in the
following way. Select the last two edges, $(d, n+i), (i+1, n+i)$ in the list and
the edge $e_j = (c, n+i)$. Now change edge $e_j$ to $(c, d)$ and remove the last
two edges in the list. Thus we have $\mathcal{O}(1)$ methods for splitting and
restoring the topology, in a way which is consistent in the numbering with the
one described in \cref{sec:representation} and shown in
\cref{fig:algorithm-topologies}.

\subsection{Order of building and optimizing}
\label{sec:order-build-optim}

When going over the implementation done by \citeauthor{smith1992} a thing that
stroke me as odd initially was the way in which \citeauthor{smith1992} keeps
track of which topology vectors has been tried, and how he choses which to try
next.

\TODO[inline]{Describe how he does this, with the stack, why it seems like a
  waste, and why it is actually VERY GOOD!}

\TODO[inline]{Naive approach when building and pruning significantly increases run
time. Explain Smiths approach with a stack sorted on the best seen/most
promising results.}

\section{Analytical solution to the Fermat-Torricelli problem}
\label{sec:analyt-solut-ferm}

\TODO[inline]{Refactor this section and move the 2D proof to appendix. Then
  instead describe the unweighted version also presented by uteshev. Afterwards
  consider either simplifying the d-space proof to unweighted or refactor it as
  well.}

An analytical solution for finding the Fermat-Torricelli point of three
prespecified points is presented by \textcite{uteshev2014}, which is the same as
finding the Steiner point of those three points. The article presents a
solution for two dimensions which will be described. However as we are working
in $\mathbb{R}^d$ a generalization of the solution to higher dimensions will
also be presented. This generalization is as far as I know, not presented
anywhere else in the literature.

The reason this is interesting is that \textcite{smith1992} mentions an
iterative process which updates points one at a time would no be as quick to
converge as the one he presents. There is however no experiments in the paper to
support this. Thus this analytical method is interesting, as it is actually not
computationally heavy to compute the location of a Steiner point, and as
each optimization moves a point to the, at that time, exact location it needs to
be to minimize the length. One could therefore think that this might converge
quite fast.

An optimization method using the analytical method has been implemented in the
new implementation. To use this instead of \citeauthor{smith1992}'s iteration
give the flag \texttt{-iteration=simple} on the command-line\footnote{as opposed
  to either having no flag or having \texttt{-iteration=smith} which both uses
  \citeauthor{smith1992}'s iteration.}.

\subsection{2D}
\label{sec:2d}

The solution for two dimensions is presented by \textcite{uteshev2014} is for
the generalized Fermat-Torricelli problem, which given three non-colinear points
$p_1 = (x_1, y_1)$, $p_2 = (x_2, y_2)$ and $p_3 = (x_3, y_3)$ in the plane,
finds the point $p_\ast = (x_\ast, y_\ast)$ which gives a solution to the
optimization problem
%
\begin{equation}
  \min_{(x,y)} F(x,y) \quad \text{for} \quad F(x,y) = \sum_{j=1}^3 m_j
  \sqrt{{(x-x_j)}^2 + {(y-y_j)}^2}
\end{equation}
%
Where the weight of point $j$th point, $m_j$, is a real positive number. The
instance where $m_1 = m_2 = m_3 = 1$ is exactly the classical Fermat-Torricelli
problem, which has a unique solution either coinciding with one of the points
$p_1$, $p_2$, $p_3$ or with the Fermat-Torricelli or Steiner point of the
triangle $p_1 p_2 p_3$. The instance with unequal weights is known as the
generalized Fermat-Torricelli point.

Existence and uniqueness of the solution is guaranteed by
\cref{thm:fermat-torricelli}
%
\begin{theorem}
\label{thm:fermat-torricelli}
  Denote the corner angles of the triangle $p_1 p_2 p_3$ by $\alpha_1, \alpha_2,
  \alpha_3$. Then if the conditions
  %
  \begin{equation}
    \label{eq:12}
    \left\{
      \begin{array}{c}
        m_1^2 < m_2^2 + m_3^2 + 2 m_2 m_3 \cos \alpha_1 , \\
        m_2^2 < m_1^2 + m_3^2 + 2 m_1 m_3 \cos \alpha_2 , \\
        m_3^2 < m_1^2 + m_2^2 + 2 m_1 m_2 \cos \alpha_3 , \\
      \end{array}
    \right.
  \end{equation}

  \TODO[inline]{Describe these equations and from where they come.}

  are fulfilled then there exists a unique solution $p_\ast = (x_\ast, y_\ast)
  \in \mathbb{R}^2$ for the generalized Fermat-Torricelli problem lying inside
  the triangle $p_1 p_2 p_3$. This point is a stationary point for the function
  $F(x,y)$, i.e.\ a real solution of the system
  \begin{equation}
    \label{eq:13}
    \sum_{j=1}^3 \frac{m_j(x-x_j)}{\sqrt{{(x-x_j)}^2 + {(y - y_j)}^2}} = 0, \quad
    \sum_{j=1}^3 \frac{m_j(y-y_j)}{\sqrt{{(x-x_j)}^2 + {(y - y_j)}^2}} = 0.
  \end{equation}

  \TODO[inline]{Describe these equations and from where they come.}

  If any of the conditions in \cref{eq:12} is violated then $F(x,y)$ attains its
  minimum value at the corresponding vertex of triangle.
\end{theorem}
%
Calculating the coordinates of the point $(x_\ast, y_\ast)$, if the conditions
in \cref{eq:12} are fulfilled, is done as follows
\begin{gather}
  \label{eq:15}
  x_\ast = \frac{K_1 K_2 K_3}{4 S \sigma d} \left( \frac{x_1}{K_1} +
  \frac{x_2}{K_2} + \frac{x_3}{K_3} \right), \quad
  y_\ast = \frac{K_1 K_2 K_3}{4 S \sigma d} \left( \frac{y_1}{K_1} +
  \frac{y_2}{K_2} + \frac{y_3}{K_3} \right)
  \intertext{with}
  \label{eq:11}
  F(x_\ast, y_\ast) = \min_{(x,y)} F(x,y) = \sqrt{d}
  \intertext{Here}
  \sigma = \frac{1}{2} \sqrt{-m_1^4 - m_2^4 - m_3^4
    + 2 m_1^2 m_2^2 + 2 m_1^2 m_3^2 + 2 m_2^2 m_3^2} \\
  \label{eq:1}
  r_{jl} = \sqrt{{(x_j - x_l)}^2 + {(y_j - y_l)}^2} = |p_j p_l|
  \quad \{j,l\} \in \{ 1, 2, 3 \} \\
  \label{eq:20}
  S = | x_1 y_2 + x_2 y_3 + x_3 y_1 - x_1 y_3 -x_3 y_2 - x_2 y_1 | \\
  \label{eq:3}
  \left\{
    \begin{array}{c}
      K_1 = (r_{12}^2 + r_{13}^2 - r_{23}^2) \sigma + (m_2^2 + m_3^2 - m_1^2) S
      \\
      K_2 = (r_{12}^2 + r_{23}^2 - r_{13}^2) \sigma + (m_1^2 + m_3^2 - m_2^2) S
      \\
      K_3 = (r_{23}^2 + r_{13}^2 - r_{12}^2) \sigma + (m_1^2 + m_2^2 - m_3^2) S
    \end{array}
  \right. \\
  d = \frac{1}{2 \sigma} (m_1^2 K_1 + m_2^2 K_2 + m_3^2 K_3) \\
\end{gather}

\begin{proof}
To prove \cref{eq:15} the first step is to prove the two following
equations\footnote{Note that all the following calculations can also be found
  in \textcite{uteshev2014}, however as these are a bit convoluted they are also
  presented with a few clarifying intermediate calculations}
%
\begin{align}
  \label{eq:16}
  K_1 K_2 + K_1 K_3 + K_2 K_3 &= 4 \sigma S d \\
  \label{eq:2}
  r_{23}^2 K_1 + r_{13}^2 K_2 + r_{12}^2 K_3 &= 2 S d
\end{align}
%
Both can proven by simply substituting the right hand sides with the definition
of each symbol, i.e.\ through direct computation. Using \cref{eq:16} we can
rewrite \cref{eq:15} as
%
\begin{gather}
  \label{eq:17}
  \left\{
    \begin{array}{c}
  x_\ast = \frac{1}{\frac{1}{K_1} + \frac{1}{K_2} + \frac{1}{K_3}} \left( \frac{x_1}{K_1} +
    \frac{x_2}{K_2} + \frac{x_3}{K_3} \right) \\
  y_\ast = \frac{1}{\frac{1}{K_1} + \frac{1}{K_2} + \frac{1}{K_3}} \left( \frac{y_1}{K_1} +
    \frac{y_2}{K_2} + \frac{y_3}{K_3} \right)
    \end{array}
  \right.
  \intertext{Which will be useful in later calculations. The second part is to
    prove that}
  \sqrt{{(x_\ast - x_j)}^2 + {(y_\ast - y_j)}^2} = \frac{m_j K_j}{2 \sigma
    \sqrt{d}}, \quad j \in \{ 1, 2, 3 \}\label{eq:18}
\end{gather}
%
As the calculations are similar for all $j$, only the one for $j = 1$ is shown
here. Thus
%
\begin{align}
  {(x_\ast - x_1)}^2
  &+ {(y_\ast - y_1)}^2 \\
  &= {\left( \frac{K_1 K_2 K_3}{4 S \sigma d} \left( \frac{x_1}{K_1} +
    \frac{x_2}{K_2} + \frac{x_3}{K_3} \right) - x_1 \right)}^2 +
    {\left( \frac{K_1 K_2 K_3}{4 S \sigma d} \left( \frac{y_1}{K_1} +
    \frac{y_2}{K_2} + \frac{y_3}{K_3} \right) - y_1 \right)}^2 \\
  &= {\left( \frac{K_1 K_2 K_3}{4 S \sigma d} \right)}^2 \left[{\left(
    \frac{x_1}{K_1} + \frac{x_2}{K_2} + \frac{x_3}{K_3} -
    \frac{x_1 4 S \sigma d}{K_1 K_2 K_3} \right)}^2 + {\left(
    \frac{y_1}{K_1} + \frac{y_2}{K_2} + \frac{y_3}{K_3} -
    \frac{y_1 4 S \sigma d}{K_1 K_2 K_3} \right)}^2 \right] \\
  &\stackrel{\mathclap{(\ref{eq:16})}}{=}
    {\left( \frac{K_1 K_2 K_3}{4 S \sigma d} \right)}^2
    \left[ {\left( \frac{x_2}{K_2} + \frac{x_3}{K_3} - \frac{x_1}{K_2} -
    \frac{x_1}{K_3} \right)}^2 + {\left( \frac{y_2}{K_2} + \frac{y_3}{K_3} -
    \frac{y_1}{K_2} - \frac{y_1}{K_3} \right)}^2 \right] \\
  &= {\left( \frac{K_1 K_2 K_3}{4 S \sigma d} \right)}^2
    \left[ \frac{{(x_2 - x_1)}^2 + {(y_2 - y_1)}^2}{K_2^2} +
    \frac{{(x_3 - x_1)}^2 + {(y_3 - y_1)}^2}{K_3^2} \right. \\
  &\hspace{7em} \left. + \; 2 \frac{(x_2 - x_1)(x_3 - x_1) +
    (y_2 - y_1)(y_3 - y_1)}{K_2 K_3} \right] \\
  &\stackrel{\mathclap{(\ref{eq:1})}}{=}
    {\left( \frac{K_1 K_2 K_3}{4 S \sigma d} \right)}^2
    \left[ \frac{r_{12}^2}{K_2^2} + \frac{r_{13}^2}{K_3^2} +
    \frac{r_{12}^2 + r_{13}^2 - r_{23}^2}{K_2 K_3} \right] \\
  &= \frac{K_1^2}{{(4 S \sigma d)}^2}
    \left[ r_{12}^2 K_3^2 + r_{13}^2 K_2^2 +
    (r_{12}^2 + r_{13}^2 - r_{23}^2) K_2 K_3 \right] \\
  &= \frac{K_1^2}{{(4 S \sigma d)}^2}
    \left[ (r_{12}^2 K_3 + r_{13}^2 K_2) (K_2 + K_3) -
    r_{23}^2 K_2 K_3 \right] \\
  &\stackrel{\mathclap{(\ref{eq:2})}}{=}
    \frac{K_1^2}{{(4 S \sigma d)}^2}
    \left[ (2 S d - r_{23}^2 K_1) (K_2 + K_3) - r_{23}^2 K_2 K_3 \right] \\
  &= \frac{K_1^2}{{(4 S \sigma d)}^2}
    \left[ 2 S d (K_2 + K_3) - r_{23}^2 (K_1 K_2 + K_1 K_3 + K_2 K_3) \right] \\
  &\stackrel{\mathclap{(\ref{eq:16})}}{=}
    \frac{K_1^2}{{(4 S \sigma d)}^2}
    \left[ 2 S d (K_2 + K_3) - 4 r_{23}^2 \sigma S d \right] \\
  &= \frac{2 S d K_1^2}{{(4 S \sigma d)}^2}
    \left[ K_2 + K_3 - 2 r_{23}^2 \sigma \right] \\
  &\stackrel{\mathclap{(\ref{eq:3})}}{=}
    \frac{K_1^2}{8 S d \sigma^2}
    \left[ 2 m_1^2 S \right] \\
  &= \frac{m_1^2 K_1^2}{4 \sigma^2 d}
\end{align}
%
To finish the proof one would also have to show that $K_1, K_2, K_3$ are
nonnegative. The proof of this can be found in~\cite[p.~5-6]{uteshev2014}.

We can now prove \cref{eq:15}, by substituting it into \cref{eq:13}. As the
equations are similar for $x$ and $y$, only $x$ is shown here
%
\begin{align}
  0
  &= \frac{m_1 (x_\ast - x_1)}{\sqrt{{(x_\ast - x_1)}^2 + {(y_\ast - y_1)}^2}} +
    \frac{m_2 (x_\ast - x_2)}{\sqrt{{(x_\ast - x_2)}^2 + {(y_\ast - y_2)}^2}} +
    \frac{m_3 (x_\ast - x_3)}{\sqrt{{(x_\ast - x_3)}^2 + {(y_\ast - y_3)}^2}} \\
  &\stackrel{\mathclap{(\ref{eq:18})}}=
    \frac{m_1 (x_\ast - x_1)}{\frac{m_1 K_1}{2 \sigma \sqrt{d}}} +
    \frac{m_2 (x_\ast - x_2)}{\frac{m_2 K_2}{2 \sigma \sqrt{d}}} +
    \frac{m_3 (x_\ast - x_3)}{\frac{m_3 K_3}{2 \sigma \sqrt{d}}} \\
  &= 2 \sigma \sqrt{d} \left[
    \frac{x_\ast - x_1}{K_1} +
    \frac{x_\ast - x_2}{K_2} +
    \frac{x_\ast - x_3}{K_3} \right] \\
  &= 2 \sigma \sqrt{d} \left[
    x_\ast \left( \frac{1}{K_1} + \frac{1}{K_2} + \frac{1}{K_3} \right) -
    \left( \frac{x_1}{K_1} + \frac{x_2}{K_2} + \frac{x_3}{K_3} \right)
    \right] \\
  &\stackrel{\mathclap{(\ref{eq:17})}}=
    2 \sigma \sqrt{d} \left[
    \frac{1}{\frac{1}{K_1} + \frac{1}{K_2} + \frac{1}{K_3}} \left( \frac{x_1}{K_1} +
    \frac{x_2}{K_2} + \frac{x_3}{K_3} \right)
    \left( \frac{1}{K_1} + \frac{1}{K_2} + \frac{1}{K_3} \right) -
    \left( \frac{x_1}{K_1} + \frac{x_2}{K_2} + \frac{x_3}{K_3} \right)
    \right] \\
  &= 2 \sigma \sqrt{d} \left[
    \frac{1}{\frac{1}{K_1} + \frac{1}{K_2} + \frac{1}{K_3}}
    \left( \frac{1}{K_1} + \frac{1}{K_2} + \frac{1}{K_3} \right) -
    1 \right] \\
  &= 2 \sigma \sqrt{d} \left[ 1 - 1 \right] = 0
\end{align}
%
As can be seen \cref{eq:15} is a solution to \cref{eq:13}. Secondly we can also
prove \cref{eq:11}
%
\begin{align}
  F(x_\ast,y_\ast)
  &= \sum_{j=1}^3 m_j \sqrt{{(x_\ast - x_j)}^2 + {(y_\ast - y_j)}^2} \\
  &\stackrel{\mathclap{(\ref{eq:18})}}{=}
    \sum_{j=1}^3 \frac{m_j^2 K_j}{2 \sigma \sqrt{d}} \\
  &\stackrel{\mathclap{(\ref{eq:20})}}{=}
    \frac{2 \sigma d}{2 \sigma \sqrt{d}} = \sqrt{d}
\end{align}
\end{proof}

\subsection{Generalization to higher dimensions}
\label{sec:gener-high-dimens}

Generalizing the solution described above to higher dimensions is in general
pretty straightforward. The biggest change is in the representation of $S$, and
the fact that the proof requires a bit of sum manipulation. This generalized
version of the proof is not known to have been presented in any paper before.

The first change is that we no longer try to minimize a function of just two
parameters, but instead we intend to minimize a function of $d$ parameters,
where our points are contained in $\mathbb{R}^d$, i.e.\ we have $d$ dimensions. As we
normally represent this using vectors, we would not actually have $d$
parameters, but instead one parameter being a $d$-vector. Thus we wish to find a
point $p_\ast = (x_{(\ast,1)}, x_{(\ast,2)}, \ldots, x_{(\ast,d)}) \in \mathbb{R}^d$
which minimizes the following function for the three points
$p_1 = (x_{(1,1)}, x_{(1,2)}, \ldots, x_{(1,d)}), p_2 = (x_{(2,1)}, x_{(2,2)},
\ldots, x_{(2,d)}), p_3 = (x_{(3,1)}, x_{(3,2)}, \ldots, x_{(3,d)}) \in \mathbb{R}^d$.
%
\begin{gather}
  \min_{(x_1, x_2, \ldots, x_d)} F(x_1, x_2, \ldots, x_d) \quad \text{for} \quad
  F(x_1, x_2, \ldots, x_d) = \sum_{j=1}^3 m_j \sqrt{\sum_{i=1}^d {(x_i -
    x_{(j,i)})}^2 }
  \\ \Updownarrow \\
  \label{eq:4}
  \min_{(p)} F(p) \quad \text{for} \quad
  F(p) = \sum_{j=1}^3 m_j | p p_j |, \quad p \in \mathbb{R}^d
\end{gather}
%
Where $|p p_j|$ is the euclidean distance between $p$ and $p_j$. The point
$p_\ast$ now no longer has to be a solution to the equation system in
\cref{eq:13}, but to a similar system containing $d$ equations, as follows
%
\begin{equation}
\label{eq:7}
  \left.
    \begin{array}{c}
    \sum_{j=1}^3 \frac{m_j(x_1-x_{(j,1)})}{| p p_j|} = 0
    \\
    \sum_{j=1}^3 \frac{m_j(x_2-x_{(j,2)})}{| p p_j|} = 0
    \\
    \vdots
    \\
    \sum_{j=1}^3 \frac{m_j(x_d-x_{(j,d)})}{| p p_j|} = 0
  \end{array}
  \right\} =
  \sum_{j=1}^3 \frac{m_j (p - p_j)}{| p p_j |} = \vec 0
\end{equation}
%
Then as before, if the conditions in \cref{eq:12} are fulfilled\footnote{If not
  the point is as before located at the corresponding vertex.} the solution to
\cref{eq:7} can be found as follows
%
\begin{gather}
  p_\ast = (x_{(\ast,1)}, x_{(\ast,2)}, \ldots, x_{(\ast,d)})\label{eq:8}
  \intertext{with}
  x_{(\ast, j)} = \frac{K_1 K_2 K_3}{4 S \sigma d} \left( \frac{x_{(1,j)}}{K_1} +
  \frac{x_{(2,j)}}{K_2} + \frac{x_{(3,j)}}{K_3} \right), \quad 0 < j \le d\label{eq:5}
  \intertext{and}
  F(p_\ast) = \min_{(p)} F(p) = \sqrt{d}\label{eq:21}
  \intertext{Here}
  \label{eq:14} r_{jl} = \sqrt{\sum_{i = 1}^d {(x_{(j,i)} - x_{(l,i)})}^2} = |p_j p_l|
  \quad \{j,l\} \in \{ 1, 2, 3 \} \\
  S = \frac{1}{2} \sqrt{(r_{12} + r_{23} + r_{13}) (r_{23} + r_{13}) (r_{12} +
    r_{13}) (r_{12} + r_{23})} \\
\end{gather}
%
And the rest of the expressions the same as in the 2D version. As can be seen,
all of the expressions, except for $S$ (and the distance which now of course
have more than two dimensions), retain the same representation as in the
two dimensional version. $S$ however looks quite different.

\TODO[inline]{Explain how these two S are the same, and maybe give Pawel credit for this one?}

\begin{proof}
As before to prove \cref{eq:5} (and thus \cref{eq:8}) the first step once again
is to prove \cref{eq:16} and \cref{eq:2}. These can as before be established
through direct computation, i.e.\ by simply substituting the expressions for
their definitions, until both sides only contain weights or distances, after
which the two sides can be seen to be equal. Using \cref{eq:16} we can rewrite
\cref{eq:5} as
%
\begin{gather}
    x_{(\ast, j)} = \frac{1}{\frac{1}{K_1} + \frac{1}{K_2} + \frac{1}{K_3}} \left( \frac{x_{(1,j)}}{K_1} +
  \frac{x_{(2,j)}}{K_2} + \frac{x_{(3,j)}}{K_3} \right), \quad 0 < j \le
d\label{eq:10}
\intertext{similarly to \cref{eq:17} in the 2D version. Secondly we wish
  to prove that}
  \sqrt{\sum_{i = 1}^d {(x_{(\ast,i)} - x_{j,i})}^2} = \frac{m_j K_j}{2 \sigma
    \sqrt{d}}, \quad j \in \{ 1, 2, 3 \}\label{eq:19}
\end{gather}
As the calculations are similar for all $j$, only the one for $j = 1$ is shown
here. Thus
%
\begin{align}
  \sum_{i=1}^d {(x_{\ast,i} - x_{1,i})}^2
  &= \sum_{i=1}^d { \left( \frac{K_1 K_2 K_3}{4 S \sigma d}
    \left( \frac{x_{(1,i)}}{K_1} + \frac{x_{(2,i)}}{K_2} + \frac{x_{(3,i)}}{K_3}
    \right) - x_{(1,i)} \right) }^2 \\
  &= { \left( \frac{K_1 K_2 K_3}{4 S \sigma d} \right) }^2 \left[ \sum_{i=1}^d
    { \left( \frac{x_{(1,i)}}{K_1} + \frac{x_{(2,i)}}{K_2} +
    \frac{x_{(3,i)}}{K_3} - \frac{x_{(1,i)} 2 S \sigma d}{K_1 K_2 K_3}
    \right) }^2 \right] \\
  &\stackrel{\mathclap{(\ref{eq:16})}}{=}
    {\left( \frac{K_1 K_2 K_3}{4 S \sigma d} \right)}^2 \left[ \sum_{i=1}^d
    {\left( \frac{x_{(2,i)}}{K_2} + \frac{x_{(3,i)}}{K_3} - \frac{x_{(1,i)}}{K_2} -
    \frac{x_{(1,i)}}{K_3} \right)}^2 \right] \\
  &= {\left( \frac{K_1 K_2 K_3}{4 S \sigma d} \right)}^2 \left[ \sum_{i=1}^d
    \left( \frac{{(x_{(2,i)} - x_{(1,i)})}^2}{K_2^2} +
    \frac{{(x_{(3,i)} - x_{(1,i)})}^2}{K_3^2} \right. \right. \\
  &\hspace{7em} \left. \left. + \; 2 \frac{(x_{(2,i)} - x_{(1,i)})(x_{(3,i)} -
    x_{(1,i)})}{K_2 K_3} \right) \right] \\
  &= {\left( \frac{K_1 K_2 K_3}{4 S \sigma d} \right)}^2
    \left[ \frac{\sum_{i=1}^d {(x_{(2,i)} - x_{(1,i)})}^2}{K_2^2} +
    \frac{\sum_{i=1}^d {(x_{(3,i)} - x_{(1,i)})}^2}{K_3^2} \right. \\
  &+ \left. \frac{ \sum_{i=1}^d {(x_{(2,i)} - x_{(1,i)})}^2 +
    \sum_{i=1}^d {(x_{(3,i)} - x_{(1,i)})}^2 -
    \sum_{i=0}^d {(x_{(2,i)} - x_{(3,i)})}^2}{K_2 K_3} \right] \\
  &\stackrel{\mathclap{(\ref{eq:14})}}{=}
    {\left( \frac{K_1 K_2 K_3}{4 S \sigma d} \right)}^2
    \left[ \frac{r_{12}^2}{K_2^2} + \frac{r_{13}^2}{K_3^2} +
    \frac{r_{12}^2 + r_{13}^2 - r_{23}^2}{K_2 K_3} \right] \\
  &= \frac{K_1^2}{{(4 S \sigma d)}^2}
    \left[ r_{12}^2 K_3^2 + r_{13}^2 K_2^2 +
    (r_{12}^2 + r_{13}^2 - r_{23}^2) K_2 K_3 \right] \\
  &= \frac{K_1^2}{{(4 S \sigma d)}^2}
    \left[ (r_{12}^2 K_3 + r_{13}^2 K_2) (K_2 + K_3) -
    r_{23}^2 K_2 K_3 \right] \\
  &\stackrel{\mathclap{(\ref{eq:2})}}{=}
    \frac{K_1^2}{{(4 S \sigma d)}^2}
    \left[ (2 S d - r_{23}^2 K_1) (K_2 + K_3) - r_{23}^2 K_2 K_3 \right] \\
  &= \frac{K_1^2}{{(4 S \sigma d)}^2}
    \left[ 2 S d (K_2 + K_3) - r_{23}^2 (K_1 K_2 + K_1 K_3 + K_2 K_3) \right] \\
  &\stackrel{\mathclap{(\ref{eq:16})}}{=}
    \frac{K_1^2}{{(4 S \sigma d)}^2}
    \left[ 2 S d (K_2 + K_3) - 4 r_{23}^2 \sigma S d \right] \\
  &= \frac{2 S d K_1^2}{{(4 S \sigma d)}^2}
    \left[ K_2 + K_3 - 2 r_{23}^2 \sigma \right] \\
  &\stackrel{\mathclap{(\ref{eq:3})}}{=}
    \frac{K_1^2}{8 S d \sigma^2}
    \left[ 2 m_1^2 S \right] \\
  &= \frac{m_1^2 K_1^2}{4 \sigma^2 d}
\end{align}
%
From here we can that \cref{eq:5} is a solution to \cref{eq:7} in the exact same way as
we proved that \cref{eq:15} was a solution to \cref{eq:13}, by substituting
\cref{eq:5} into \cref{eq:7} and using \cref{eq:10} and \cref{eq:19}. The same
goes for proving \cref{eq:21}.
\end{proof}

\section{Sorting points}
\label{sec:sorting-points}

\TODO[inline]{Describe the existing methods used with sorting points, and give a
reason/explanation for the one tried here. Possibly also discuss/suggest other
alternative such as e.g.\ picking the first two that are farthest apart.}

Sorting the terminals in one way or the other before starting to build
topologies and optimizing them has been proposed by \textcite{smith1992} and at
least two different kinds of sorting has been implemented by
\textcite{fonseca2014,vanlaarhoven2013}. Both methods have shown promising
results regarding the number of trees one needs to optimize actually optimize in
contrast to not sorting the terminals. Here another method is
proposed which is relatively similar to the one proposed and implemented
in~\cite{fonseca2014}, but simpler and requiring fewer steps.

The method proposed in~\cite{fonseca2014} corresponds to solving the problem of
finding the longest path in a fully connected graph, where the graph consists of
all the terminals. This problem however is NP-hard\missingref{}~and thus
it might be an idea instead to approximate this in a greedy manner.

The first terminal remains the same as the unsorted one. All following terminals
are selected between the remaining terminals as such that the distance to the
prior point is maximized. Thus consider e.g.\ the list of one dimensional points
$[1, 2, 3, 4, 5, 6]$. These would be sorted as $[1, 6, 2, 5, 3, 4]$.

Sorting the points using the described method can be done by passing the flag
\texttt{-sort} on the command-line.

%%% Local Variables:
%%% mode: latex
%%% TeX-master: "../../main"
%%% End:
