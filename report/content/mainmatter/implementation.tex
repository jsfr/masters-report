 {
\abnormalparskip{0pt}
\chapter{Implementation}
\label{cha:implementation}
}

This chapter will describe the architecture of the new implementation, i.e.\ how
the new implementation has been structured.

The new implementation has been written in the, still relatively new but at this
point stable, language Go, which quoting the homepage: ``Go is an open source
programming language that makes it easy to build simple, reliable, and efficient
software.''~\cite{golanghomepage}. Go is a general-purpose, strongly typed and
garbage-collected language. The syntax bears resemblance to C\footnote{But
  with a lot less braces, and no pointer arithmetic.}. Furthermore Go has
explicit support for concurrent programming. The language furthermore sports
extremely fast compiling of source code and tries to have performance
comparable to C.

The reasoning behind choosing Go for the new implementation were firstly that I
already had experience with the language making it easy to get to work on the
new implementation without also having to learn a new programming
language. Secondly the language has, in my opinion, a very clean syntax which
makes it easily readable for anyone familiar with any C-syntax styled
language. Finally the easy use of concurrency in Go\footnote{Starting a new
  goroutine, which is the languages type of lightweight threads, is as simple as
  writing: \texttt{go~SomeFunction~(args\ldots)}.} made it interesting as
continued work with the new implementation could include a potential,
significant, speedup by making the branching of the algorithm concurrent. The
same could be be true for the Gauss elimination, concurrent\footnote{Not
  concurrency has actually been implemented in the new implementation. However
  the possibilities of this is discussed in \cref{sec:concurrency}.}.

In general the implementation tries to follow guidelines given in
\textcite{effectivego}. These are guidelines as to how one codes effective
Go. Here effective means both in terms of speed, memory usage, stability and
readability.

Source code for the new implementation, fixes for the old implementation and
experiments can be found at \url{https://github.com/jsfr/SteinerExact}. The
source code for just the new implementation is also included in
\cref{cha:source-code}.

\section{Overview}
\label{sec:overview-1}

In general the code is structured into a package (or library) called
\texttt{smt} and a main file (which is actually split into a main and
configuration file). This is done to allow for easy reuse of the main amount of
work, as it is a library.

The \texttt{smt} package consists of all the data structures and and their
functions related to calculating \acp{smt}. The main file consists mainly of the
main function which mainly reads the flags from the command-line and then starts
the main loop---Which iterates through generated topologies, similarly to
\citeauthor{smith1992}'s main loop---with the correct flags, i.e.\ whether we
should use sorting, which iteration we should use, etc. Finally the
configuration file consists of a data structure and function for reading the
command-line flags---this is what is used by the main function for this.

\section{\acs{smt} Package}
\label{sec:smt-package}

As Go is an imperative language we do not have classes, objects, instances and
so on as in e.g.\ Java. We do however have structs similar to $C$, on which we
can also define functions. Thus the \texttt{smt} package is structured such that
most of the files of it is named similar to the struct and functions on it it
holds. E.g.\ the file \texttt{smt/edge.go} this files contains the struct
defining an edge, and the methods on this, such as calculating the length of the
edge.

\subsection{Structs}
\label{sec:structs}

The most important structs for the new implementation are:
%
\begin{itemize}
\item \textbf{Point} \quad A struct which just represent a $d$ dimensional
  point. Implemented as a simple array of floats. For convenience points has a
  number of functions for subtracting points, calculating dot products, and
  finally the function for sorting terminals, which is simply implemented as a
  function that takes a list of points and sorts them as described in \cref{sec:sorting-terminals}.
\item \textbf{Edge} \quad A struct representing an edge of a tree. The edge
  consists of a pointer to the tree and two integers which are the indicies of
  the end points of the edge in the point-array of the tree. The pointer is
  needed of for the indicies to make sense, and to easily calculate the length
  of the edge.
\item \textbf{Tree} \quad A struct represent a complete tree. The struct
  contains a list of all edges, a list of points, the dimension of the the
  points and the number of points which are terminals. Finally the struct also
  contains a list of adjacencies to the Steiner points, as this was considered
  the easiest way to hold these, instead of having to find them every time they
  are needed in the edges. The most important functions (besides the iterations)
  on at tree are those for splitting and restoring and edge and the function for
  calculating the error of the tree.
\item \textbf{Stack} \quad The implementation of \citeauthor{smith1992}'s
  iteration utilizes quite a few stacks. These were in the original
  implementation just arrays on which \citeauthor{smith1992} made sure to always
  index as if it was a stack. To ensure that the stacks are always used
  correctly, the new implementation instead implements a simple stack struct,
  which just consists of an array as the underlying structure, and then
  functions for pushing and popping elements of the stack.
\end{itemize}

\subsection{Helpers}
\label{sec:helpers}

The \texttt{smt} package also contains a number of helpers. These are function
for calculating the pertubed centroid of three points, calculating the
Fermat-Torricelli point of three points, printing a trees representation to the
console. Only the function for printing a tree to console can be accessed
outside the package. The others are for internal use only.

\subsection{Iterations}
\label{sec:iterations-1}

Finally the package also contains a file with an implementation of the two
different iterations. The first one is the iteration described by
\textcite{smith1992} (named \texttt{SmithsIteration}) and the second is a
iteration based of the simple iteration described in \cref{sec:simple-iteration}
using the analytical solution for the Fermat-Torricelli problem presented by
\textcite{uteshev2014} to place the Steiner points (named \texttt{SimpleIteration}).

The second iteration is relatively simple in its iteration. It simple iterates
through all Steiner points, calculating their placement at with regards to the
other points current placement.

The first iteration in general follows the same outline as done in the original
implementation. In the same way it also utilizes a number of arrays and stacks
which are global to the package (but inaccessible from the outside to avoid
someone using the library from tampering with the data mid-iteration
unintentionally). In general these global arrays is unwanted as it clutters the
space of the other functions in the package. However initial runs of the
iteration, showed that the new implementation used a significant amount of time
in the iteration on creating new local arrays every time it was being run. This
was of course undesirable, as it both wasted time, and made comparison to the
original code harder. Thus after some consideration, and for lack of a better
solution, I decided to continue using global variables for the arrays and stacks
of \texttt{SmithsIteration}.

\section{Main and Configuration File}
\label{sec:main-file}

Finally the new implementation contains a main file, and a configuration file.

The main file contains the main function being run when the program is executed
on the command-line, and the main loop for enumerating and optimizing
topologies. This is done using the structs and functions of the \texttt{smt}
package, i.e.\ its public API. The main loop in general follow the same outline
as the main loop of \citeauthor{smith1992}'s original implementation.

The main file furthermore uses the functions and structs of the configuration
file for reading and handling the arguments given on the command-line. The
implementation unlike \citeauthor{smith1992}'s implementation does not read a
list of terminals on STDIN, but instead takes as argument a file-path to what it
expects to be a JSON-file containing a key ``points'' which should be a list of
equal length lists, corresponding to the terminals. The syntax of the file is
the standard JSON syntax

\section{Other Source Code and Files}
\label{sec:other-source-code}

code not related to the new implementation

experiments an their helper scripts

the fixed version of smiths original code

the geosteiner package

%%% Local Variables:
%%% mode: latex
%%% TeX-master: "../../main"
%%% End:
