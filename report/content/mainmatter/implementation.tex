 {
\abnormalparskip{0pt}
\chapter{Implementation}
\label{cha:implementation}
}

This chapter touches on parts of the implementation which are done differently
from Smiths implementation.

The new implementation is written in Googles, still relatively new, language
Golang~\cite{GolangHomepage}

\section{Overview}
\label{sec:overview-1}

\section{Building topologies}
\label{sec:building-topologies}

Naive approach when building and pruning significantly increases run
time. Explain Smiths approach with a stack sorted on the best seen/most
promising results. This cuts from ~10s to ~0.5s.

\section{Analytical solution to the Fermat-Torricelli problem}
\label{sec:analyt-solut-ferm}

An analytical solution for finding the Fermat-Torricelli point of three
prespecified points is presented in~\cite{Uteshev2012}, which is the same as
finding the Steiner point of those three points.  The article presents a
solution for two dimensions which will be described in \cref{sec:2d} and
afterwards I will generalize the solution to higher dimensions
in \cref{sec:gener-high-dimens}.

\TODO[inline]{Why is this interesting/how has it been used in the code?}

\subsection{2D}
\label{sec:2d}

The solution for two dimensions is presented in~\cite{Uteshev2012}.  The solution
presented is for the generalized Fermat-Torricelli problem, which given three
noncolinear points $P_1 = (x_1, y_1)$, $P_2 = (x_2, y_2)$ and $P_3 = (x_3, y_3)$
in the plane, finds the point $P_\ast = (x_\ast, y_\ast)$ which gives a solution
to the optimization problem
%
\begin{equation}
  \label{eq:11}
  \min_{(x,y)} F(x,y) \quad \text{for} \quad F(x,y) = \sum_{j=1}^3 m_j
  \sqrt{{(x-x_j)}^2 + {(y-y_j)}^2}
\end{equation}
%
Where the weight of point $j$th point, $m_j$, is a real positive number.  The
instance where $m_1 = m_2 = m_3 = 1$ is exactly the classical Fermat-Torricelli
problem, which has a unique solution either coinciding with one of the points
$P_1$, $P_2$, $P_3$ or with the Fermat-Torricelli or Steiner point of the
triangle $P_1 P_2 P_3$.  The instance with unequal weights is known as the
generalized Fermat-Torricelli point.

Existence and uniqueness of the solution is guaranteed by
\cref{thm:fermat-torricelli}
%
\begin{theorem}
  Denote the corner angles of the triangle $P_1 P_2 P_3$ by $\alpha_1, \alpha_2,
  \alpha_3$. Then if the conditions
  \begin{equation}
    \label{eq:12}
    \left\{
      \begin{array}{c}
        m_1^2 < m_2^2 + m_3^2 + 2 m_2 m_3 \cos \alpha_1 , \\
        m_2^2 < m_1^2 + m_3^2 + 2 m_1 m_3 \cos \alpha_2 , \\
        m_3^2 < m_1^2 + m_2^2 + 2 m_1 m_2 \cos \alpha_3 , \\
      \end{array}
    \right.
  \end{equation}
  are fulfilled then there exists a unique solution $P_\ast = (x_\ast, y_\ast)
  \in \mathbb{R}^2$ for the generalized Fermat-Torricelli problem lying inside
  the triangle $P_1 P_2 P_3$.  This point is a stationary point for the function
  $F(x,y)$, i.e.\ a real solution of the system
  \begin{equation}
    \label{eq:13}
    \sum_{j=1}^3 \frac{m_j(x-x_j)}{\sqrt{{(x-x_j)}^2 + {(y - y_j)}^2}} = 0, \quad
    \sum_{j=1}^3 \frac{m_j(y-y_j)}{\sqrt{{(x-x_j)}^2 + {(y - y_j)}^2}} = 0.
  \end{equation}
  If any of the conditions in \cref{eq:12} is violated then $F(x,y)$ attains its
  minimum value at the corresponding vertex of triangle.
\end{theorem}
%


\subsection{Generalization to higher dimensions}
\label{sec:gener-high-dimens}

\section{Sorting points}
\label{sec:sorting-points}

longest path in fully connected graph is NP-complete [ref]. Greedy approach
shows promising results.

\chapterbreak{}

%%% Local Variables:
%%% mode: latex
%%% TeX-master: "../../main"
%%% End:
