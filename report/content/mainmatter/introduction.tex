{
\abnormalparskip{0pt}
\chapter{Introduction}
\label{cha:introduction}
}

\TODO[inline]{Write introduction}

\section{Objectives}
\label{sec:objectives}

The objective of the project will first and foremost be to implement Smiths
algorithm for finding Steiner Minimal Trees in Euclidean d-Space efficiently. In
the original paper by Smith the proposed C implementation has a lot of
double work and ineffiecency and thus it is the main objective of this project
to implement this algorithm effiecently and then benchmark this against the
original implementation.

A secondary objective of the project will be to explore and propose possible
improvements to the algorithm. Some of which might be explored are:

* Sorting the terminals in  different orders before finding the positions.
* Analytically solving the problem with fewer (3) terminals at a time to improve
  limits where the Steiner points may move very slowly normally.
* An analysis of the stop-criteria used by Smith to see whether this is actually
  correct, and whether there are possible improvements.

If there is time the project will also look at versions of the problem such as
the fixed-orientation version, where edges are only allowed to have certain
predefined orientations.

The learning goals of the project are:

* Understanding, analysing and describing complex algorithms and solutions to a
  specific problem area
    - In this case the problem of finding Steiner Minimal Trees
      in Euclidean d-Space using Smiths algorithm)
* On the basis of the analysis propose possible improvements to the existing
  solution, implement and test the improvements.
* Benchmarking

\section{Structural Outline}
\label{sec:structural-outline}

To provide the reader with an overview of what is to come, an
outline of the different chapters of the thesis is presented here:

\begin{itemize}
\item \textbf{Chapter 2: Preliminaries} \quad
\item \textbf{Chapter 3: \citeauthor{smith1992}'s Algorithm} \quad 
\item \textbf{Chapter 4: Data Structures and Methods} \quad
\item \textbf{Chapter 5: Implementation} \quad
\item \textbf{Chapter 6: Experiments} \quad
\item \textbf{Chapter 7: Discussion} \quad
\item \textbf{Chapter 8: Conclusion} \quad
\end{itemize}

%%% Local Variables:
%%% mode: latex
%%% TeX-master: "../../main"
%%% End:
