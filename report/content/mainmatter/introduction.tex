{ \abnormalparskip{0pt}
  \chapter{Introduction}
  \label{cha:introduction}
}

The \acl{estp} is the problem of finding the \acl{smt} for some given set of
points (known as terminals). The \acl{smt} is the tree containing these
terminals, and having the shortest possible length. The tree is allowed to
contain extra points (known as Steiner points) if adding these to interconnect
the terminals can reduce the length of the tree.

Apart from being a problem that is interesting to solve simply because it is
hard, the problem is also known to have application in designing electronic
circuits and a novel approach for phylogenetics have also been suggested by
\textcite{brazil2009}.

The problem is known to be NP-hard, however in two dimensions efficient
algorithms for finding \aclp{smt} exists, some of which can solve instances with
thousands of terminals. In many cases these can solve the problem much quicker
than the worst-case running time. These algorithms however, do not generalize to
higher dimensions, and as such we need to find other methods. The most
well-known of these is the one originally presented by \textcite{smith1992}.
However the algorithm is not able to solve instances with more than around
15--20 terminals. The reason for this lies with the fact that the algorithm
must enumerate all possible full Steiner topologies (apart from those it is able
to prune).

There have been articles to propose extensions on the algorithm to further push
the boundaries of solvable instances. These have included sorting of terminals,
using lower bounds, and pruning via geometric conditions. None of these have
pushed the boundaries for solvable instances with more than a few terminals.

It is therefore still interesting to explore possible improvements of the
algorithm, either by a more efficient implementation, extensions, or by changing
parts of the algorithm (such as the iteration for optimizing a tree).

The thesis propose improvements (these have also been implemented in a
new implementation) to the implementation of \citeauthor{smith1992}'s algorithm
given in \textcite{smith1992}. Furthermore a new simple iteration based on the
analytical solution for the Fermat-Torricelli problem proposed by
\textcite{uteshev2014} is proposed and implemented. The thesis then presents and
discusses experiments performed to test the correctness and efficiency (compared
to the original implementation) of the new implementation.

The thesis also presents and discuss elements of the article and implementation
in \textcite{smith1992} which are questionable.

Apart from this the thesis collects and presents preliminaries to gain an
understanding of the \acl{estp} and the current state of the research in the
field.

Finally the thesis proposes a number of possible directions in which one could
continue the work of this thesis to further improve on the new implementation.

\section{Objectives}
\label{sec:objectives}

A overview of primary and secondary objectives is provided here. The primary
objectives are:
%
\begin{itemize}
\item Understand, analyze and describe the algorithm for finding \aclp{smt} in
  euclidean $d$-space, presented by \textcite{smith1992}. A sub-objective of
  this is also to identify and discuss elements of the article which are
  questionable (e.g.\ the proposed error-function). It also includes identifying
  and discussing double-work and other questionable elements of the C-implementation
  given by \textcite{smith1992}.
\item Propose a new and possibly more efficient data structure and method for building
  topologies instead of the one used by \citeauthor{smith1992}'s implementation.
  Then implement the algorithm presented in \textcite{smith1992}, using the new
  data structure.
\item Present the analytical solution for the Fermat-Torricelli problem for $3$
  points presented for two dimensions by \textcite{uteshev2014} and generalize
  it to higher dimensions. Then propose and implement a simple algorithm for
  optimizing full Steiner topologies using the analytical solution. This is to
  be seen as a alternative to the algorithm given by \textcite{smith1992}.
\item Describe existing methods for ordering the terminals when processed by the
  the algorithm. Then propose and implement a new method for ordering terminals.
\item Perform experiments to test the correctness and efficiency of the new
  implementation and compare it with the original implementation.
\end{itemize}

The secondary objectives are:
%
\begin{itemize}
\item Present preliminaries (definitions, proofs, etc.) for understanding
  Steiner trees and topologies in euclidean $d$-space.
\item Present methods for finding \aclp{smt} in two dimensions, and why these do not
  generalize to higher dimensions.
\item Discuss possible extensions for the new implementation (e.g.\
  concurrency).
\end{itemize}

\section{Structural Outline}
\label{sec:structural-outline}

To provide the reader with an overview of what is to come, an outline of the
different chapters of the thesis is presented here:

\begin{itemize}
\item \textbf{Chapter 2: Preliminaries} \quad presents the \acl{estp}. The
  chapter also presents the different types of Steiner trees and topologies. The
  chapter afterwards presents some of the most well-known methods for finding
  Steiner trees in 2D, why they are effective and discusses why these do not
  generalize to higher dimensions. The chapter then presents some of the known
  related work into the field of finding Steiner trees in $d$-space. Finally the
  chapter presents some variations and applications of the Steiner tree in
  $d$-space, to get a feel of possible uses for a solution to the problem.
\item \textbf{Chapter 3: \citeauthor{smith1992}'s Algorithm} \quad presents the
  algorithm for finding \aclp{smt} in euclidean $d$-space, as proposed by
  \textcite{smith1992}. The chapter also present a simple iteration for
  optimizing \aclp{fst}. Finally the chapter discuss some of the more
  questionable elements of the implementation done by \citeauthor{smith1992}.
\item \textbf{Chapter 4: Data Structures and Methods} \quad presents the improvements and changes made by the new implementation. The
  chapter first presents the new data structure and method proposed for building
  topologies, which will be used instead of the original implementation's which
  was to just rebuild the entire tree from scratch every time. The chapter then
  presents and generalizes to higher dimensions an analytical solution
  (originally presented by \textcite{uteshev2014}) for solving the
  Fermat-Torricelli problem. This is presented here, as it is used as the
  optimization iteration in conjunction with the simple iteration presented in
  Chapter 3.
\item \textbf{Chapter 5: Implementation} \quad presents the new
  implementation, its structure and gives a general overview of how the
  architecture of the new implementation is set up. The chapter also reasons for
  the choice of programming language.
\item \textbf{Chapter 6: Experiments} \quad presents the
  experimental work done to test the new implementation. The experimental work
  falls into two categories: correctness and performance. The chapter first
  presents work on correctness, and describes the sub-optimality observed with
  the implemented simple iteration in some instances. The chapter furthermore
  describes the work done to identify the cause of the sub-optimality, and
  discusses the possible reasons on the basis of this work. The chapter
  afterwards present the work done to test the performance of the implementation
  compared to the original implementation. The performance is measured by three
  different parameters: speed/run-time, number of optimized trees and number of
  iterations. The chapter describes and discusses the observed behavior of the
  two implemented optimization-iterations in comparison to the original
  implementation. This is done with and without terminal sorting.
\item \textbf{Chapter 7: Discussion} \quad discusses possible
  extensions and improvements to the work of the thesis, most of which are
  related to the implementation (e.g.\ concurrency and how one could do this).
\item \textbf{Chapter 8: Conclusion} \quad summarizes the various
  results and conclusions from the thesis. The suggested improvements and the
  implementation is evaluated.
\end{itemize}

%%% Local Variables:
%%% mode: latex
%%% TeX-master: "../../main"
%%% End:
