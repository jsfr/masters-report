{
\abnormalparskip{0pt}
\chapter{Introduction}
\label{cha:introduction}
}

In two dimensions efficient algorithms for finding \aclp{smt} exists, some of
which can solve instances with thousands of terminals. These algorithms, however do
not generalize to higher dimensions, and as such we need to find other methods.
The most well-known of these is the one originally presented by
\textcite{smith1992}. However the algorithm is not able to solve instances with
more than around $15--20$ terminals.

There have been articles to propose extensions on the algorithm to further push
the boundaries of solvable instances. These have included sorting of terminals,
using lower bounds, and pruning via geometric conditions. None of these have
pushed the boundaries for solvable instances with more than a few terminals.

It is therefore still interesting to explore possible improvements of the
algorithm, either by a more efficient implementation, extensions, or by changing
parts of the algorithm (such as the iteration for optimizing a tree).

\section{Objectives}
\label{sec:objectives}

To provide the reader with an overview of the primary and secondary goals of
this thesis. The primary are:
%
\begin{itemize}
\item Understand, analyze and describe the algorithm for finding \aclp{smt} in
  euclidean $d$-space, presented by \textcite{smith1992}. A sub-objective of
  this is also to identify and discuss elements of the article which are
  questionable (e.g.\ the proposed error-function). It also includes identifying
  and discussing double-work, or questionable elements of the C-implementation
  given by \textcite{smith1992} (e.g.\ the way in which topologies are built).
\item Propose a, possibly more efficient, data structure and method for building
  topologies instead of the one used by \citeauthor{smith1992}'s implementation.
  Then implement the algorithm presented in \textcite{smith1992}, using the new
  data structure.
\item Present and generalize to $d$ dimensions an the analytical solution for
  the Fermat-Torricelli problem presented for two dimensions by
  \textcite{uteshev2014}. Then propose and implement a simple algorithm for
  optimizing full Steiner topologies using the analytical solution. This is to
  be seen as a alternative to the algorithm given by \textcite{smith1992}.
\item Describe existing methods for sorting terminals. Then propose and
  implement a new greedy method for sorting terminals.
\item Perform experiments to test the correctness of the new implementation and
  compare it with the old implementation.
\end{itemize}

The secondary are:
%
\begin{itemize}
\item Present preliminaries (definitions, proofs, etc.) for understanding
  Steiner trees and topologies in euclidean $d$-space.
\item Present methods for finding \aclp{smt} in two dimensions, and why
  these do generalize to higher dimensions.
\item Discuss possible extensions for the new implementation (e.g.\
  concurrency).
\end{itemize}

\section{Structural Outline}
\label{sec:structural-outline}

To provide the reader with an overview of what is to come, an
outline of the different chapters of the thesis is presented here:

\begin{itemize}
\item \textbf{Chapter 2: Preliminaries} \quad This chapter presents the
  \acl{estp} which is the problem this thesis is about solving. The chapter also
  presents the different types of Steiner trees and topologies. The presentation
  includes. The chapter afterwards presents some of the most well-known methods
  for finding Steiner trees in 2D, why they are effective and discusses why these
  do not generalize to higher dimensions. The chapter then presents some of
  the known related work into the field of finding Steiner trees in $d$-space.
  Finally the chapter presents some variations and applications of the Steiner
  tree in $d$-space, to get a feel of possible uses for a solution to the
  problem.
\item \textbf{Chapter 3: \citeauthor{smith1992}'s Algorithm} \quad This chapter
  presents the different parts that go into the combined algorithm as present by
  \textcite{smith1992}. Initially the representation and generation of
  topologies is described. The reasoning behind the representation is also
  given, by describing how we can use the representation to prune topologies
  which cannot become the \acl{smt}. A simple iteration which can be used to
  optimize the tree of a given topology is then presented. This is done to both
  approach the iteration proposed by \citeauthor{smith1992} somewhat softer, and
  as this iteration will be used later in the thesis. The chapter then presents
  the iteration proposed by \citeauthor{smith1992} for optimizing the tree of a
  given topology. Finally the chapter presents and discusses elements of the
  article by \textcite{smith1992} that I found questionable or odd. This
  includes both elements of the article and the actual implementation he gives.
\item \textbf{Chapter 4: Data Structures and Methods} \quad This chapter
  presents the possible improvements and changes made by the new implementation.
  The chapter first presents the new data structure and method proposed for
  building topologies, which will be used in stead of the original
  implementation's which was to just rebuild the entire tree from scratch every
  time. The chapter then presents and generalizes to higher dimensions an
  analytical solution (originally presented by \textcite{uteshev2014}) for
  solving the Fermat-Torricelli problem. This is presented here, as it is used
  as the optimization iteration in conjunction with the simple iteration
  presented in Chapter 3.
\item \textbf{Chapter 5: Implementation} \quad This chapter presents the new
  implementation, its structure and gives a general overview of how the
  architecture of the new implementation is set up. The also chapter reasons for
  the choice of programming language.
\item \textbf{Chapter 6: Experiments} \quad This chapter presents the
  experimental work done to test the new implementation. The experiment work
  falls in two categories: correctness and performance. The chapter first
  presents work on correctness, and describes the sub-optimality observed with
  the implemented simple iteration in some instances. The chapter furthermore
  describes the work done to identify the cause of the sub-optimality, and
  discusses the possible reasons on the basis of this work. The chapter
  afterwards present the work done to test the performance of the implementation
  compared to the original implementation. The performance is measured by three
  different parameters: speed/run-time, number of optimized trees and number of
  iterations. The chapter describes and discusses the observed behavior of the
  two implemented optimization iterations in comparison to the original
  implementation. This is done with and without terminal sorting. 
\item \textbf{Chapter 7: Discussion} \quad This chapter discusses some of the
  observations made and obstacles met during the work of this thesis. The
  chapter also discusses possible extensions and improvements to the work of the
  thesis, most of which are related to the implementation (e.g.\ concurrency and
  how one could do this).
\item \textbf{Chapter 8: Conclusion} \quad This chapter summarizes the various
  results and conclusions from the thesis. The suggested improvements and the
  implementation is evaluated.
\end{itemize}

%%% Local Variables:
%%% mode: latex
%%% TeX-master: "../../main"
%%% End:
