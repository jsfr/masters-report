{
\abnormalparskip{0pt}
\chapter{Preliminaries}
\label{cha:preliminaries}
}

% Short introduction to the chapter (max 1/2 page)
The following chapter introduces basic concepts and definitions which either
help in understanding the problem area of the thesis, or which are directly used
by the thesis. The most important keywords of this chapter, which are directly
used in the thesis, are:
%
\begin{itemize}
\item Topologies and trees
\item Steiner trees: \aclp{fst}, \aclp{smt}
\item \aclp{estp}
\item Fermat-Torricelli point/problem
\end{itemize}
%
At the same time this chapter will introduce versions of the Steiner tree
problem which are not used directly by the thesis, but which help to give an
understanding of the problem area of the thesis, and which are mentioned in
\cref{cha:introduction} or discussed in \cref{cha:discussion}.

This chapter is mostly based on~\textcite{smith1992,gilbert1968,brazil2015} and
will by large follow the structure of~\cite[ch.~1]{brazil2015}.

\section{The Fermat-Torricelli Problem}
\label{sec:ferm-torr-probl}

Before introducing the Steiner problem, it is relevant to introduce the
Fermat-Torricelli problem, as this can be seen as a sort of subproblem to be
solved when solving the Steiner tree problem.

\newpage

The Fermat-Torricelli problem\footnote{The problem is so named as it was first
  proposed by \textcite{fermat1891} and the earliest known solution was put
  forth by \textcite{torricelli1919}.} in it classical two dimensonal form, is
defined as follows:
%
\begin{center}
\begin{tabular}{rp{9cm}}
  \toprule
  \textbf{Given} & A set of three points $V = \{p_1, p_2, p_3\}$ lying in the plane. \\
  \textbf{Find}  & A point $s$ such that the sum of the Euclidean distances from
                   $s$ to $p_1, p_2$ and $p_3$ is minimized. \\
  \bottomrule
\end{tabular}
\end{center}
%
The point $s$ will be referred to as a Steiner point\footnote{In the
  Fermat-Torricelli problem this point would normally be named the Fermat point
  or the Fermat-Torricelli point. However for clarity as to the connection
  between this problem and the Steiner problem this term is used. The name of
  both the Steiner problem/tree/point is named after Jakob Steiner, which might
  be errornous, who supposedly studied the Steiner problem for $3$
  terminals~\cite{brazil2014}.}. There are several ways to solve this problem,
e.g.\ using the rotation-proof~\cite[p.~3--5]{brazil2015}.

This formulation of the problem can be seen as a specialized instance of the
more general problem, where the edges are weighted. In the classical version the
edge weights are all equal to each other.

The generalized version for two dimensions of the problem can be formulated as in
\textcite{uteshev2014}:
%
\begin{center}
  \begin{tabular}{rp{9cm}}
    \toprule
    \textbf{Given} & Three non-colinear points $p_1 = (x_1, y_1)$, $p_2 = (x_2,
                     y_2)$ and $p_3 = (x_3, y_3)$ in the plane. \\
    \textbf{Find} & The point $p_\ast = (x_\ast, y_\ast)$ which gives a solution
                    to the optimization problem
                    \begin{gather}
                      \min_{(x,y)} F(x,y) \quad \text{for} \\
                      F(x,y) = \sum_{j=1}^3 m_j \sqrt{{(x-x_j)}^2 + {(y-y_j)}^2}
                    \end{gather}
    Where the weight of $j$th point, $m_j$, is a real positive number. \\
    \bottomrule
  \end{tabular}
\end{center}
%
The generalized Fermat-Torricelli problem can be even further generalized from
two dimensions to $d$ dimensions, with $d \ge 2$
\cite{fermattorricelliproblem}. This simply requires the points to be
$d$-dimensional, and we then change the optimization problem to the following:
%
\begin{gather}
  \min_{(x_1, x_2, \ldots, x_d)} F(x_1, x_2, \ldots, x_d) \quad \text{for} \\
  F(x_1, x_2, \ldots, x_d) = \sum_{j=1}^3 m_j
  \sqrt{\sum_{i=1}^d {(x_i - x_{(j,i)})}^2 }
  \\ \Updownarrow \\
  \label{eq:4}
  \min_{(p)} F(p) \quad \text{for} \quad
  F(p) = \sum_{j=1}^3 m_j | p p_j |, \quad p \in \mathbb{R}^d
\end{gather}
%
Solving the Fermat-Torricelli problem with even weights is equivalent with solving
the smallest possible instance of the \acl{estp}. The analytical solution
presented by~\textcite{uteshev2014} and a generalization of it to $\mathbb{R}^d$ is
presented in \cref{sec:analyt-solut-ferm}.

\section{The Euclidean Steiner Tree Problem}
\label{sec:eucl-stein-tree}

We firstly define a graph $G = (V(G), E(G))$ where the vertices $V(G)$ are
points of the graph, $V(G) \subset \mathbb{R}^d$ and the edges
$E(G) \subset (\mathbb{Z}^{+} \times \mathbb{Z}^{+})$ are straight lines
connecting the points in $V(G)$. The euclidean length of edge $e \in E(G)$ we
denote $|e|$.

The \ac{estp} is then defined as follows:
%
\begin{center}
  \begin{tabular}{rp{9cm}}
    \toprule
    \textbf{Given} & A set of points $R = \{ p_1, p_2, \ldots, p_n \}$ in
                     $\mathbb{R}^d$. \\
    \textbf{Find} & A graph $T = (V(T), E(T))$ such that $R \subseteq V(T)$, and
                    $|T| = \sum_{e \in E(T)} |e|$ is minimized. \\
    \bottomrule
  \end{tabular}
\end{center}
%
Note that $T$ must be a tree, as any cycles can obviously be removed
without disconnecting the graph or increasing the length.

\section{Steiner Minimal Tree}
\label{sec:steiner-minimal-tree}

We define a \ac{smt} as in \textcite{brazil2015}, which is as follows:
%
\begin{definition}[\acl{smt}, Steiner points, terminals]
  A tree $T = (V(T), E(T))$ representing a solution to the Steiner tree problem
  is called a \acl{smt}. The given points $R \subseteq V(T)$ are called
  terminals and possible extra vertices $S = V(T) \setminus R$ are called
  Steiner points.
\end{definition}
%
The definition used by \textcite{smith1992} is simply that a \ac{smt} on $n$
points is the shortest tree containing the points. This definition, while being
a bit more convoluted, is essentially the same as the one given by
\textcite{brazil2015}. Thus we use that definition as it is more verbose and
gives a definition of Steiner points and terminals as well.

\section{Topologies}
\label{sec:topologies-1}

A topology is the combinatorial structure of a tree, or other form of geometric
network. This means that in a topology only the adjacencies of points
matter. Thus we could represent the topology of the tree $G = (V(G), E(G))$ as
just the edges of the tree $E(G)$. Thus any graph (and tree) has an
underlying topology.

We call any tree of some topology non-degenerate if all edges of the tree have
non-zero length---otherwise the tree is called degenerate.

We now define a Steiner topology, and a full Steiner topology as the following
%
\begin{definition}[Steiner topology, full Steiner topology]
The topology of a non-degenerate \ac{smt} is called a Steiner topology. The
topology of a non-degenerate \ac{smt} in which every terminal has degree 1 is
called a full Steiner topology.
\end{definition}
%
Notice that a full Steiner topology must necessarily have the most possible
Steiner points so $ k = n - 2$. This can be seen by substituting $n_2 = n_3 = 0$
and $n_1 = n$ in \cref{eq:22}.

\section{Steiner trees}
\label{sec:steiner-trees}

Before defining Steiner trees we define \acp{rmt} as in~\cite{gilbert1968}
as:
%
\begin{definition}[\acl{rmt}]
  A tree is called a \acl{rmt} if it is the shortest possible tree of its
  underlying topology.
\end{definition}
%
As can easily be seen a \ac{smt} must also be a \ac{rmt} of its underlying
topology. Finally we define Steiner trees. Steiner trees currently have two
definitions which are not exactly identical.

The first is the definition used by \textcite{gilbert1968}, combined with the
definition for \acp{fst} in \textcite{smith1992}, is as follows:
%
\begin{definition}[Steiner tree, \acl{fst}]
  If a tree can be shortened no further even when splitting is allowed, the tree
  is called a Steiner tree. If all of the original points\footnote{I.e.\ the
    points not inserted when splitting. These are what we have so far referred
    to as the terminals.} of a Steiner tree have degree 1 the tree is called a
  \acl{fst}.\label{def:steiner-tree}
\end{definition}
%
Here the process of splitting referred to is exactly what we do when e.g.\
inserting an extra point in the Fermat-Torricelli problem. Thus it follows
pretty straightforward that a Steiner tree must also be a \ac{rmt} of the
topology which describes it\footnote{Note that here we refer to the topology we
  end up with after we cannot split any more points, and not the topology we
  started out with.}. Furthermore any \ac{smt} must also be a Steiner tree, but
not vice versa as illustrated by \cref{fig:steiner-smts}.
%
\begin{figure}[htbp]
\centering
\includegraphics[width=0.6\textwidth]{gfx/tikz/steiner-smts}
\caption[Example of splitting and \acp{rmt}]{Example of how of splitting a point
  and inserting a new Steiner point. The flow shows two of the paths splitting
  could take on the shown topology down until no more splitting is
  possible. Optimizing the two final topologies, both are obviously \acp{rmt} of
  their underlying topology, but only the right one is a \ac{smt} as the other
  topology is degenerate (the two Steiner points lie on top of each other in the
  center). Thus if the original points are the terminals, only the last right
  tree is a \ac{smt} for the original terminals.\label{fig:steiner-smts}}
\end{figure}
%
The other, and newer definition, used by \textcite{brazil2015} is as follows:
%
\begin{definition}[Steiner tree, \acl{fst}]
  A non-degenerate (full) \ac{rmt} for a Steiner topology is called a (full)
  Steiner tree.\label{def:steiner-tree2}
\end{definition}
%
These definitions are not exactly identical as e.g.\ the last tree to the left
in \cref{fig:steiner-smts} would not be a Steiner tree by
\cref{def:steiner-tree2}\footnote{As it has an edge of length zero.}  but would
be by \cref{def:steiner-tree}. The similarities of the two definitions should
however be obvious, and it also holds for both of the definitions that any
\ac{smt} must be a Steiner tree, and that any Steiner tree must be a \ac{rmt} of
its underlying topology.

Finally there is also the definition used by
\textcite{smith1992}, which defines a Steiner tree, by some properties which is
must also have for it to satisfy the above definitions. This is as follows
%
\begin{definition}[Steiner tree]
\leavevmode\vspace{-\baselineskip}\par
\begin{enumerate}
\item It contains $n$ terminals
  $R = \{ p_1, p_2, \ldots, p_n \} \in \mathbb{R}^d$, and possibly $k$
  additional Steiner points
  $S = \{ p_{n+1}, \ldots, p_{n+k} \} \in \mathbb{R}^d$.
\item Each Steiner point has degree $3$, the edges emanating from it are
  coplanar and have a mutual angles of $120^{\circ}$.
\item Each terminal has degree at most $3$.
\item $0 \le k \le n-2$.
\end{enumerate}
\end{definition}
%
\begin{proof}
The first property is by definition, to name the terminals and the
Steiner points.

The second and third property can be shown in the following way: First consider
that any Steiner point must have at least degree $3$. This should be obvious as
a Steiner point with degree $1$ does not connect any terminals to the rest of
the tree, and thus the point can simply be removed to either shorten the tree or
keep it the same\footnote{Which could happen if a Steiner point was lying atop
  the point is connected to.}. Furthermore every Steiner point with a degree of
$2$ can be removed, and its edges be replaced with one edge that is of the same
length or shorter as per the Triangle Inequality
Theorem\cite{triangleinequality}.

We now show that no pair of edges emanating from a Steiner point can have a
mutual angle less than $120^{\circ}$. This can proved in a few different
ways. The one used by \textcite{gilbert1968} is based on the mechanical model
presented described in the same article. A simpler approach in my own opinion
however is the one used by \textcite{brazil2015}. The proof is follows: Consider
a point $a$ in $T$, and a pair of non-zero edges $ab, ac \in E(T)$ meeting at
$a$. Note that $ab$ and $ac$ must form a shortest interconnection of the points
$\{a, b, c\}$ and furthermore we know that a solution the Fermat-Torricelli
problem for $\{ a, b, c \}$ forms a shortest interconnection as well. An
immediate consequence\footnote{This stems from the fact that the
  Fermat-Torricelli problem has a solution at one of the points if any of the
  angles is greater than $2 \pi / 3$ and somewhere between them otherwise. The
  details can either be found in \textcite[ch.~1]{brazil2015} or somewhat
  implicitly in \cref{sec:analyt-solut-ferm}.} of this is that the angle between
$ab$ and $ac$, denoted $\angle bac$ must be at least $2 \pi / 3 = 120^{\circ}$.
To see that this is true, assume that it is smaller; We then have two cases as
shown in \cref{fig:preliminaries-steiner-point} and in neither cases is $ab$ and
$ac$ a shortest interconnection of $\{a, b, c\}$.
%
\afterpage{
\begin{figure}[htbp]
\centering
\includegraphics[width=0.6\textwidth]{gfx/tikz/preliminaries-steiner-points}
\caption[Obtaining shorter trees]{Replacing a pair of edges $ab$ and $ac$ for
  which $\angle bac < 2 \pi / 3$ with a shortest interconnection provided by a
  solution to the Fermat-Torricelli problem for $\{a,b,c\}$. The left figure has
  no angle greater than $120^{\circ}$. The right has a angle
  $\angle cba \ge 120^{\circ}$. In both cases a shorter tree is
  constructed\footnotemark.\label{fig:preliminaries-steiner-point}}
\end{figure}
\footnotetext{This figure is the same as in \textcite[p.~7]{brazil2015}.}}
%
Using that the mutual angles may be no less than $120^{\circ}$ it follows that
any point in $T$ can have at most degree $3$\footnote{As
  $3 \cdot 120^{\circ} = 360^{\circ}$ meaning that there is at most room for
  three edges around a point.}. It therefore follows that terminals have a
degree of at most $3$, which proves the third property, and that Steiner points
have a degree of exactly $3$ and thus also mutual angles of precisely
$120^{\circ}$, proving the second property.

The fourth property follows in the following way from property 3 and 4. Any tree
$T$ has $|V(T)|-1$ edges. Thus a Steiner tree has $n+k-1$ edges.  Every edge has
two endpoints meaning there are $2 n + 2 k - 2$ edges. As every Steiner point
has a degree of $3$, they account for $3 k$ of the endpoints. For the terminals
we must split them in three groups: $n_1$, those that have a degree of
$1$. $n_2$, those that have a degree of $2$ and, $n_3$, those that have a degree
of $3$. The terminals thus account for the rest of the endpoints, written as
$n_1 + 2 n_2 + 3 n_3 \ge n$. We can then write the equation as
%
\begin{align}
  \label{eq:22}
  3 k + n_1 + 2 n_2 + 3 n_3 &= 2 n + 2 k - 2 \\
  k &= 2 n - 2 - (n_1 + 2 n_2 + 3 n_3) \\
  0 \le k &\le n - 2
\end{align}
%
Thus proving the fourth property.
\end{proof}

\section{Finding Steiner Trees in 2D}
\label{sec:find-stein-trees-2}

In general the problem of finding \acp{smt} is NP-hard. This however is only a
measure of the worst-case upper bound. Thus in some cases it may indeed be
possible to find them much faster.

In two dimension algorithms exists, which in many cases can the \acp{smt} much
faster than the worst-case scenario.

This section will give a short introduction to the Hwang-Melzak and GeoSteiner
algorithm. Afterwards the section will discuss why these algorithms do not
generalize to dimensions higher than two.

\subsection{The Hwang-Melzak Algorithm}
\label{sec:hwang-melz-algor}

The Hwang-Melzak algorithm was first presented by \textcite{melzak1961} and
later improved by \textcite{hwang1986hexagonal}. For a proof of correctness see
\textcite{hwang1986linear, melzak1961}.

The algorithm is used for optimizing the tree $T$ of a pre-specified full
Steiner topology $mathcal{T}$ with at least $n \ge 5$ terminals, assuming that
$T$ is non-degenerate. The algorithm can be implemented to run in
$\mathcal{O}(n)$. The algorithm is very well-known, and is also presented in
\textcite{brazil2015,smith1992}.

The algorithm consists of two steps. The first, the \textit{merging} step, works
as follows: Let $u$ be a Steiner point connected to $a$, $b$ and $s$, where $a$
and $b$ are terminals. Replace $a$, $b$ and $u$ with a the equilateral point
$e_{ab}$\footnote{An equilateral point $c$, of the points $a$ and $b$, is a
  point such that the triangle $\triangle a b c$ is equilateral. It is clear
  that in two dimensions such a point has two unique placements, one on each
  side of the line passing through $a$ and $b$.} of $a$ and $b$. Do this $n-2$
times until the topology consists of only $2$ points. The second step, the
\textit{reconstruction} step, works a follows: Let $\mathcal{T}'$ be the
full Steiner topology on the $n-1$ new terminals\footnote{I.e.\ the same set of terminals as
  before, but where $a$ and $b$ has been replace with $e_{ab}$}. Then the
coordinates of $u$ can be found by taking the intersection between the Steiner
arc $\hat{ab}$\footnote{The arc between $a$ and $b$ on the circle circumscribing
  the triangle $\triangle a b e_{ab}$.} and the edge $e_{ab}s$. Do this on the
$2$ point tree $n-2$, reinserting the points represented by the equilateral
points, until we again have a tree of $n$ terminals with the correct coordinates
for the $n-2$ Steiner points.

The biggest issue of the above algorithm is deciding on which side of the line
$ab$ we should place the equilateral point $e_{ab}$. However it turns out that
by choosing the order in which we perform the merging steps we are always able
to find the correct side. Thus this algorithm runs in $\mathcal{O}(n)$.

The proof of the algorithm and how to decide the side for the equilateral point
can be found in \textcite[ch.~1]{brazil2015}.

The algorithm unfortunately does not generalize from two dimensions to higher
dimensions. The reason for this is, that while the edges emanating from a
Steiner point are indeed coplanar, the plane the points lie in is not known in
advance if there are more than $3$ terminals. This means that we no longer have
a unique pair of equilateral points for each cherry, but instead a continuous
range of equilateral points lying on a circle in $d$-space, as illustrated by
\cref{fig:equi-points}.
%
\begin{figure}[htbp]
  \centering
  \begin{subfigure}[t]{0.4\textwidth}
    \includegraphics[width=\textwidth]{gfx/tikz/equi-points-2d}
    \caption{When $d = 2$ we have exactly two unique possible
      placements.\label{fig:equi-points-2d}}
  \end{subfigure}\hspace{1em}%
  \begin{subfigure}[t]{0.4\textwidth}
    \includegraphics[width=\textwidth]{gfx/tikz/equi-points-3d}
    \caption{When $d \ge 3$ we have a circle in $d$-space of possible
      placements. Here $d = 3$ is shown.\label{fig:equi-points-3d}}
  \end{subfigure}
  \caption[Equilateral points in 2D and $d$-space]{The case where $d = 2$ only
    has the two unique possible placements of the equilateral point $e_{ab}$ for the
    two points $a$ and $b$, one of which can be eliminated when running the
    algorithm. However when $d \ge 3$ this is no longer the case, and we are
    unable to determine a unique placement of the equilateral
    point.\label{fig:equi-points}}
\end{figure}
%
Thus in contrast to the two dimensional case where we can select one of the two
unique equilateral points, in higher dimensions we are no longer able to do
this, and the algorithm breaks down.

\subsection{The GeoSteiner Algorithm}
\label{sec:geosteiner-algorithm}

The GeoSteiner algorithm was originally proposed by \textcite{winter1985}. Later
improvements of the implementation have allowed for the computation of \acp{smt}
for several thousand terminals~\cite{brazil2015}.

The GeoSteiner algorithm is pretty involved, and will thus not be described in
greater detail here. In general the algorithm has two phases. In the first
(\textit{generation}) phase, \acp{fst} for all full Steiner topologies of all
subsets of terminals are generated. This enumeration can be said to be done
implicitly\footnote{as opposed to explicitly defining each \ac{fst} one at a
  time.} by simulating the Hwang-Melzak algorithm, and using geometric
properties to prune parts of the Steiner arcs on which the Steiner points can
lie. In this way the algorithm can prune \acp{fst} of full Steiner which cannot
be part of the final \ac{smt}, and if all \acp{fst} of a full Steiner topology
is pruned, i.e.\ if no feasible part of a Steiner arc is left, the entire full
Steiner topology can be pruned. The strength of the algorithm lies partly in
generating \acp{fst} with similar full Steiner topologies in a single pass and
partly in being able to prune partially constructed full Steiner topologies. In
the second, (\textit{concatenation}) phase, the algorithm selects a subset of
not pruned \acp{fst} that span all terminals and has the minimum total
length. This concatenation can be formulated as the minimum spanning tree
problem in a hypergraph, which is indeed also NP-hard. However using a
branch-and-cut algorithm most instances can be solved quite
effectively\footnote{For an implementation of GeoSteiner see e.g.\
  \url{http://www.diku.dk/hjemmesider/ansatte/martinz/geosteiner/}.}. For a
fuller description of the algorithm, see e.g.\ \textcite[sec.~1.4]{brazil2015}.

While the algorithm is the most effective currently known for $d = 2$, it is
unfortunately not applicable when $d \ge 3$. First of, the generation of
\acp{fst} for all subsets of terminals in $\mathbb{R}^d, d \ge 3$, is much more
difficult than in $\mathbb{R}^2$. Optimizing a \ac{fst} with more than $3$
terminals when $d \ge 3$ requires solving high-degree
polynomials~\cite{smith1992}. As a consequence of this we can never use a
algorithm such as Hwang-Melzak, but must use numerical approaches, such as
\citeauthor{smith1992}'s optimization routine described in
\cref{cha:algorithm}. Such numerical approaches seem to block the generation of
\acp{fst} across various subsets of terminals. Finally many of the geometrical
properties used to prune seems to be much weaker when
$d \ge 3$~\cite{fonseca2014}.

Finally, as described by \textcite{smith1992}, it seems that when $d \ge 3$ the
topology of a \ac{smt} often consists of only a single large \ac{fst}. Thus we
will not be able to leverage the benefits of the two phases in GeoSteiner, as
the \acp{fst} we generate in the first phase will span the entire set of
terminals, making that phase very large and the second phase trivial.

\section{Finding Steiner Trees in d-space}
\label{sec:find-stein-trees-d}

As described in \cref{sec:geosteiner-algorithm} determining \acp{fst} in
$\mathbb{R}^d, d \ge 3$ requires solving high-degree polynomials, and thus we
need to resort to numerical approaches.

The most well-known of these is \citeauthor{smith1992}'s algorithm, proposed by
\textcite{smith1992}. This algorithm is described in much greater detail in
\cref{cha:algorithm}, but in general it works by enumerating all full Steiner
topologies and then finding then optimizes these using an iteration which
converges to the \ac{fst} of that full Steiner topology. The algorithm can be
implemented with a Branch-and-Bound approach, such that the found \acp{fst} can
be used to prune full Steiner topologies before optimizing them. The algorithm
however cannot solve problem instances with more than around $15$ to $20$
terminals within a feasible amount of time.

There have been several attempts at optimizing \citeauthor{smith1992}'s
algorithm to improve its speed and thus make higher number of terminals feasible
to solve. Some of these are

\begin{itemize}
\item \textcite{fampa2008} proposed using lower bounds on partial Steiner
  topologies\footnote{I.e.\ topologies that do not include all $n$ terminals.}
  to prune \acp{fst}. Roughly the method works in the following way: Imagine we
  have a \ac{fst} not connecting all $n$ terminals. To select the next terminal
  to connect to the topology we compute the lower bounds
  $z^{\ast}(\bar{D}_l^{+})$ of every terminal not yet connected, and select the
  terminal for which the largest number of descending topologies can be
  pruned. In this way the lower bound allow \citeauthor{fampa2008} to both prune
  some descendants of topologies, and also vary the order in which they connect
  terminals to reduce the number of topologies they need to optimize. The conic
  formulation and the actual lower bound will not be discussed here as they are
  rather extensive. The method to shows some improvement on both the number of
  topologies enumerated and the actual running time. However as pointed out by
  \textcite{fonseca2014} while a smaller number of \acp{fst} are generated,
  the time spent on computations increase significantly, and thus the speed
  gained from computing these lower bounds is minimal when compared to e.g.\
  distance-based sorting of the terminals.
\item \textcite{vanlaarhoven2013} proposed both a method for sorting the
  terminals by their distance to the centroid and a set of geometric criteria
  based on the lune-properties combined with the bottleneck distances used to
  prune non-optimal \acp{fst}. The use of geometric criteria is partly what
  makes the 2D algorithms so successful, and thus it seems natural to explore
  these for the $d$-space case as well. However again as noted by
  \textcite{fonseca2014} these computations give very little in when compared to
  simply sorting the terminals based on distance in terms of speed vs.\ time
  spent on computations.
\item \textcite{fonseca2014} a way of sorting the terminals, such that the first
  three terminals maximized the sum of their pairwise distance, and all terminals
  afterwards maximize the sum of their distances to the currently sorted
  terminals.
\end{itemize}

\PAWEL{\textcite{fampa2014}
  (\url{http://www.iasi.cnr.it/aussois/web/uploads/2015/papers/fampam.pdf}) der
  benytter en MINLP formulering af ESTP kan også være relevant, men så vidt jeg
  kan se er den ikke publiceret nogen steder endnu. Kan jeg tage den med?}

Finally \textcite{fonseca2014} has also proposed a new branch enumeration
algorithm, instead of \citeauthor{smith1992}'s algorithm. This algorithm draws
its inspiration from the GeoSteiner algorithm. Instead enumerating full Steiner
topologies and optimizing their respective trees it creates branches of subsets
of terminals. The algorithm utilizes that upon removed a Steiner point from a
full Steiner topology, $3$ branches are created, and that one can always find a
Steiner point, such that the branches have at most $\lfloor \frac{n}{2} \rfloor$
terminals, which reduces the number of branches that need to be generated. Tee
algorithm then consists of three phases. First it computes \acp{smt} for subsets
with up to $8$ terminals. Second it generates branches containing up to
$\lfloor \frac{n}{2} \rfloor$ terminals, and the \acp{smt} found in the first
phase are used to prune away branches that cannot be part of the final
\ac{smt}. Finally it generates full Steiner topologies with $n$ terminals by
concatenating three disjoint branches. The shortest tree found is then
outputted. \citeauthor{fonseca2014} noted that the branch enumeration algorithm
was able to solve many of the tested topologies with $15$ terminals within the
given time of 12 hours which \citeauthor{smith1992}'s, even with terminal
sorting, was not.

There has also been some work into heuristics and approximation algorithms. This
is however somewhat limited, and out of scope for this thesis as we here mainly
focus on \citeauthor{smith1992}'s algorithm and exact algorithms.

\section{Variations on the Steiner Tree Problem and Applications}
\label{sec:vari-stein-tree}

Steiner trees can of course be modified in several different ways. One could
think of changing both the metric, the constraints, the cost function and so
on. Some of the more well-known, are presented here with reference to where one
could read more about the type of Steiner tree if so desired. The section also
discuss some of the real world applications of Steiner trees, i.e.\ their
usefulness outside the academic world as more than a mathematical curiosity.

Some of the types of Steiner trees to mention are:
%
\begin{itemize}
\item \textbf{Rectilinear Steiner Tree} \quad A version of Steiner tree in which
  the Euclidean $\mathcal{L}_2$ norm is exchanged with the rectilinear
  $\mathcal{L}_1$ norm as the used metric. This means that edges consist of
  vertical and horizontal line segments. See e.g.~\textcite[ch.~3]{brazil2015}.
\item \textbf{Fixed Orientation Steiner Tree} \quad A version of Steiner tree,
  which can be seen as a generalization of the rectilinear Steiner tree. In this
  version the metric used is a fixed orientation metric, meaning that edges are
  only allowed to have some fixed orientations, the rectilinear version can be
  seen as a fixed orientation version where we have two perpendicular
  orientations. See e.g.~\textcite[ch.~2]{brazil2015}.
\item \textbf{Prize Collecting Steiner Tree} \quad A version of Steiner tree in
  which edges are ascribed some cost\footnote{This could e.g.\ be the Euclidean
    length of the edge} and points are ascribed some prize. The tree seeks to
  interconnect all terminals, while minimizing the edge cost sum \textit{and}
  maximize the prize of Steiner points. See e.g.~\textcite{johnson2000}.
\end{itemize}
%
Of possible applications, some more mentionable are:
%
\begin{itemize}
\item \textbf{Electronic circuits/VLSI Design} \quad Both rectilinear and fixed
  orientation Steiner trees have use in VLSI design and the realization of
  electronic circuits as described by
  \textcite[sec.~2.7,sec~3.6]{brazil2015}. When designing electronic circuits
  the wires between components can normally only be drawn in a rectilinear, or
  lately fixed orientation, matter. Thus Steiner trees can be used to pack the
  components closer. This also has uses for higher dimensions than $2$. As
  electronic components normally consists of several layers in which the wires
  can run between, thus crossing each other without shorting, the problem also
  has use in $3$ dimensions.
\item \textbf{Object detection} \quad A method for object detection using
  Steiner trees has been proposed by \textcite{russakovsky2010}. The method
  proposed utilizes a directed Steiner tree by showing that one can reduce the
  parameter selection\footnote{The parameter selection is roughly used to figure
    out how much of a computation can be shared between object classes. This
    preferable as the steps of object detection are computationally
    expensive. For a more detailed description of object detection,
    see~\cite{russakovsky2010}.} of the object detection to the problem of
  finding the Steiner tree in a directed graph. The problem of finding a
  directed Steiner tree as described by \textcite{russakovsky2010}, is a bit
  different from the way we have defined Steiner trees so far. Here we have a
  directed graph $G = (E(G), V(G))$, with weights $w(E)$ on the edges, a set of
  terminals $R \subseteq V(G)$ and a root node $r \in V(G)$. The goal is then to
  find a tree in $G$, rooted at $r$, that spans all terminals in $R$ while
  minimizing the sum of weights on the used edges. Thus we are not able to place
  the Steiner points ``freely'' as we have described so far, but must instead
  use the points of the graph $G$. The proposed method have shown a promising
  speedup, with a small drop in accuracy, when compared to other methods within
  that field.
\item \textbf{Phylogenetics} \quad A method using $d$-space Euclidean Steiner trees
  has been proposed to be used in the determination of phylogenetic trees by
  \textcite{brazil2009}. Phylogenetic trees are used to determine the
  relationship between different species, these are represented as the leafs of
  the tree, and the interior nodes are then their ancestors. In this way one can
  represent the genetic difference of species, as their distance in the
  phylogenetic tree.
\end{itemize}
%
\textcite{brazil2015} cover several more possible variations on Steiner trees,
and applications of Steiner trees for the interested reader.

%%% Local Variables:
%%% mode: latex
%%% TeX-master: "../../main"
%%% End:
