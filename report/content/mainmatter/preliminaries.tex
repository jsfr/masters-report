{
\abnormalparskip{0pt}
\chapter{Preliminaries}
\label{cha:preliminaries}
}

% Short introduction to the chapter (max 1/2 page)
The following chapter introduces basic concepts and definitions which either
help in understanding the problem area of the thesis, or which are directly used
by the thesis. The most important keywords of this chapter, which are directly
used in the thesis, are:
%
\begin{itemize}
\item Topologies and trees
\item Steiner trees: \acrlongpl{fst}, \acrlongpl{smt}
\item \acrlongpl{estp}
\item Fermat-Torricelli point/problem
\end{itemize}
%
At the same time this chapter will introduce versions of the Steiner tree
problem which are not used directly by the thesis, but which helps to give an
understanding of the problem area of the thesis, and which are mentioned in
\cref{cha:introduction} or discussed in \cref{cha:discussion}.

This chapter is mostly based on~\textcite{smith1992,gilbert1968,brazil2015} and
will by large follow the structure of~\cite[ch.~1]{brazil2015}.

\section{The Fermat-Torricelli Problem}
\label{sec:ferm-torr-probl}

Before introducing the Steiner problem, it is relevant to introduce the
Fermat-Torricelli problem, as this can be seen as a sort of subproblem to be
solved when solving the Steiner tree problem.

The Fermat-Torricelli problem\footnote{The problem is so named as it was first
  proposed by \textcite{fermat1891} and the earliest known solution was put
  forth by \textcite{torricelli1919}.} in it classical two dimensonal form, is
defined as follows:
%
\begin{center}
\begin{tabular}{rp{9cm}}
  \toprule
  \textbf{Given} & A set of three points $V = \{P_1, P_2, P_3\}$ lying in the plane. \\
  \textbf{Find}  & A point $s$ such that the sum of the Euclidean distances from
                   $s$ to $P_1, P_2$ and $P_3$ is minimized. \\
  \bottomrule
\end{tabular}
\end{center}
%
The point $s$ will be referred to as a Steiner point\footnote{In the
  Fermat-Torricelli problem this point would normally be named the Fermat point
  or the Fermat-Torricelli point. However for clarity as to the connection
  between this problem and the Steiner problem this term is used. The name of
  both the Steiner problem/tree/point is named after [missingref] in which he
  studied the Steiner problem for $n = 3$, $n$ being the number of
  terminals.}\missingref{Jacob Steiner}. There are several ways to solve this problem, e.g.\ using the
rotation-proof~\cite[p.~3--5]{brazil2015}.

This formulation of the problem can be seen as a specialized instance of the
more general problem, where the edges are weighted. In the classical version the
edge weights are all equal to each other.

The generalized version for two dimensions of the problem can be formulated as in
\textcite{uteshev2014}:
%
\begin{center}
  \begin{tabular}{rp{9cm}}
    \toprule
    \textbf{Given} & Three noncolinear points $P_1 = (x_1, y_1)$, $P_2 = (x_2,
                     y_2)$ and $P_3 = (x_3, y_3)$ in the plane. \\
    \textbf{Find} & The point $P_\ast = (x_\ast, y_\ast)$ which gives a solution
                    to the optimization problem
                    \begin{gather}
                      \min_{(x,y)} F(x,y) \quad \text{for} \\
                      F(x,y) = \sum_{j=1}^3 m_j \sqrt{{(x-x_j)}^2 + {(y-y_j)}^2}
                    \end{gather}
    Where the weight of $j$th point, $m_j$, is a real positive number. \\
    \bottomrule
  \end{tabular}
\end{center}
%
The generalized Fermat-Torricelli problem can be even further generalized from
two dimensions to $d$ dimensions, with $d \ge 2$
\cite{fermattorricelliproblem}. This simply requires the points to be
$d$-dimensional, and we then change the optimization problem to the following:
%
\begin{gather}
  \min_{(x_1, x_2, \ldots, x_d)} F(x_1, x_2, \ldots, x_d) \quad \text{for} \\
  F(x_1, x_2, \ldots, x_d) = \sum_{j=1}^3 m_j
  \sqrt{\sum_{i=1}^d {(x_i - x_{(j,i)})}^2 }
  \\ \Updownarrow \\
  \label{eq:4}
  \min_{(P)} F(P) \quad \text{for} \quad
  F(P) = \sum_{j=1}^3 m_j | P P_j |, \quad P \in \Re^d
\end{gather}
%
Solving the Fermat-Torricelli problem with even weights corresponds to solving
the smallest possible instance of the \acrlong{estp}. The analytical solution
presented by~\textcite{uteshev2014} and a generalization of it to $\Re^d$ is
presented in \cref{sec:analyt-solut-ferm}.

\section{The Euclidean Steiner Tree Problem}
\label{sec:eucl-stein-tree}

We firstly define a graph $G = (V(G), E(G))$ where the vertices $V(G)$ are
points of the graph, $p \in V(G) \land p \in \Re^d$ and the edges $E(G)$ are
straight lines connecting the points in $V(G)$. The euclidean length of edge $e \in
E(G)$ we denote $|e|$.

The \gls{estp} is then defined as follows:
%
\begin{center}
  \begin{tabular}{rp{9cm}}
    \toprule
    \textbf{Given} & A set of points $R = \{ p_1, p_2, \ldots, p_n \}$ in
                     $\Re^d$. \\
    \textbf{Find} & A graph $T = (V(T), E(T))$ such that $R \subseteq V(T)$, and
                    $|T| = \sum_{e \in E(T)} |e|$ is minimized. \\
    \bottomrule
  \end{tabular}
\end{center}
%
Note that $T$ must be a tree, as any cycles can obviously be removed
without disconnecting the graph or increasing the length.

\section{Steiner Minimal Tree}
\label{sec:steiner-minimal-tree}

A \gls{smt} can be defined either as in~\textcite{brazil2015} or
in~\textcite{smith1992}. These are equivalent and the the second can be proven
to follow from the first. Thus we here present the first as the definition, and
show how the properties which define the second follow. Thus:
%
\begin{definition}[\acrlong{smt}, Steiner points]
  A tree $T = (V(T), E(T))$ representing a solution to the Steiner tree problem
  is called a \acrlong{smt}. The given points $R \subseteq V(T)$ are called
  terminals and possible extra vertices $S = V(T) \setminus R$ are called
  Steiner points.
\end{definition}
%
It turns out that that a \gls{smt} has the following four properties which can
also be used a definition:
%
\begin{enumerate}
\item It contains $n$ ``terminals'' $\vec{x}_1 \cdots \vec{x}_n$, and
 possibly $k$ additional ``Steiner points'' $\vec{x}_{n+1} \cdots
 \vec{x}_{n+k}$.
\item Each Steiner point has valence $3$, and the edges emanating from it lie in a
 plane and have mutual angle $120^{\circ}$
\item Each terminal has valence between $1$ and $3$ (and generally $\le 2$).
\item $0 \le k \le n-2$.
\end{enumerate}
%
The first property is by definition, to name the terminals and the
Steiner points.

The second property can be shown in the following way: First of consider that
any Steiner point must have at least valence $3$. This should be obvious as a
Steiner point with valence $1$ does not connect any terminals to the rest of the
tree, and thus the point can simply be removed to either shorten the tree or
keep it the same\footnote{Which could happen if a Steiner point was lying atop
  the point is is connected to.}. Furthermore every Steiner point with a valence
of $2$ can be removed, and its edges be replaced with one edge that is of the
same length or shorter as per the Triangle Inequality
Theorem\cite{triangleinequality}.

% We now show that no pair of edges emanating from a Steiner point can have a
% mutual angle less than $120^{\circ}$. Suppose the contrary is the true, and the
% lines $PR$ and $RQ$ meet with angle $PRQ < 120^{\circ}$. If we use the
% mechanical interpretation given by Gilbert and Pollak~\cite{gilbert1968} the two
% edges ``pull'' on point $R$ with a resultant force of magnitude\NOTE{I don't
%  quite get why this holds true?}
% %
% \begin{equation}
% F = 2 \cos \theta / 2 > 1
% \end{equation}
% %
% \TODO[inline]{Finish the part about the angles being no less than 120}

% Using that the mutual angles may be no less than $120^{\circ}$ it follows that
% every Steiner point can have a valence of at most $3$, as
% $3 \cdot 120^{\circ} = 360^{\circ}$. We therefore have, using the first part about the
% valence that a Steiner point always has a valence of $3$, and as the mutual angles
% can be no less than $120^{\circ}$, that the angles between edges emanating from
% a Steiner point must be exactly $120^{\circ}$.

% \TODO[inline]{Write proof of third property}

% The fourth property follows in the following way from property 3 and 4. Any
% tree has one edge fewer than vertices. Thus a Steiner tree has $N+K-1$ edges.
% Every edge has two endpoints meaning there are $2N+2K-2$ edges. As every
% Steiner point has a valence of $3$, they account for $3K$ of the endpoints. For
% the terminals we must split them in three groups -- $N_1$, those that have a
% valence of $1$. $N_2$, those that have a valence of $2$ and, $N_3$, those that
% have a valence of $3$. The terminals thus account for the rest of the
% endpoints, written as $N_1 + 2 N_2 + 3 N_3 \ge N$. We can then write the
% equation as
% %
% \begin{align}
%  3K + N_1 + 2N_2 + 3N_3 &= 2N + 2K - 2 \\
%  K &= 2N - 2 - (N_1 + 2N_2 + 3N_3) \\
%  0 \le K &\le N - 2
% \end{align}



% Secondly we define a \acrlong{rmt} as follows:
% %
% \begin{definition}[\acrlong{rmt}]
%   A tree $T$ is called a \gls{rmt} if its
% \end{definition}
% %
% Finally we define Steiner trees as in \textcite{gilbert1968}:
% %
% \begin{definition}[Steiner tree]
%  A tree is called a Steiner tree if it cannot be shortened by a small perturbation or by
% ``splitting'' a terminal and inserting a Steiner point.
% \end{definition}

% Here a Steiner point is
% an extra vertex that may be inserted into the graph and used as a ``connection
% point'' for edges, as can be seen in \cref{fig:preliminaries-steiner-point}.
% %
% \begin{figure}[htbp]
% \centering
% \includegraphics[width=0.5\textwidth]{gfx/tikz/preliminaries-steiner-points}
% \caption[Steiner point of isosceles triangle.]{The length of the tree on the left
%  can be shortened by inserting an extra point as has been done on the
%  right.\label{fig:preliminaries-steiner-point}}
% \end{figure}
% %
% Another definition, used by Smith~\cite{smith1992}, defines the Steiner tree as
% tree with the following properties
% %

% \subsection{Full Steiner tree}
% \label{sec:full-steiner-tree}

% A \gls{fst} is a Steiner tree where all terminals have valence 1, which is
% equivalent to the tree having exactly $N-2$ Steiner points, the maximal number
% of possible Steiner points. This can easily be seen by considering the
% definition of Steiner trees in \cref{sec:steiner-tree}. This corresponds to the
% proof of the fourth property, but here $N_1 = N$ and $N_2 = N_3 = 0$, meaning
% that the number of Steiner points will be exactly $N-2$.

% It can furthermore be shown that any Steiner tree is a union of edge-disjoint
% \glspl{fst}, meaning that any edge of a \gls{smt} joining two terminals is
% also an edge of \gls{mst} of the terminals. This is relatively easily seen,
% by realizing that every three points connected to a Steiner point is a \gls{fst}
% for those three points. These are then either connected to the rest of the tree
% with a Steiner point, in which case they are part of a bigger \gls{fst}, or they
% are connected to a terminal, in which case we can we can remove that edge
% to have one \gls{fst} and the rest of the tree.

% The topology of a \gls{fst} is called a Full Steiner Topology, also abbreviated
% \gls{fst}. Here we have the fixed coordinates of the terminals, but not the
% coordinates of the Steiner points, and as such we only think about the
% adjacencies of the points. Whether we are talking about the topology or tree
% will be clear from the context.

% \subsection{Steiner minimal tree}
% \label{sec:steiner-minimal-tree}

% A minimal tree on $N$ points $\vec{x}_{1},\vec{x}_{2},\ldots,\vec{x}_{n}$ is the
% tree having these points as its vertices and having the smallest possible sum of
% the length of all its edges. When we are allowed to insert any number of extra
% points into the tree and connect the edges using these to further shorten the
% tree this is known as a \gls{smt}~\cite{gilbert1968}.

% It can be proven that all \glspl{smt} are Steiner trees~\cite{smith1992}.

\section{Topology}
\label{sec:topology}

A topology is the combinatorial structure of a tree, or other form of geometric
network. This means that in a topology only the adjacencies of points
matter. Thus we could represent the topology of the tree $G = (V(G), E(G))$ as
just the edges of the tree $T(T) = E(G)$.

\section{Other versions of the Steiner tree problem}
\label{sec:other-vers-stein}

\TODO[inline]{Write about some of the other versions of the problem, e.g.\ fixed
 direction, and rectilinear. Maybe L1 metric.}

\chapterbreak{}

%%% Local Variables:
%%% mode: latex
%%% TeX-master: "../../main"
%%% End:
