{
\abnormalparskip{0pt}
\chapter{Preliminaries}
\label{cha:preliminaries}
}

% Short introduction to the chapter (max 1/2 page)
This chapter contains all the preliminaries, basic concepts and definitions
that are needed to understand the work done in the thesis, and that are
not necessarily known to the reader in full.

\NOTE[inline]{This can be a bit more verbose.}

\section{Topology}
\label{sec:topology}

The combinatorial structure of a tree, or other form of geometric network, is
called a topology.  I.e.\ in the topology we only think about the adjacencies of
the points and not their coordinates.

\section{Steiner tree}
\label{sec:steiner-tree}

At least two definitions of a Steiner tree exists. These are worded
differently, bu are indeed equivalent.

The first definition, used by Gilbert and Pollak~\cite{Gilbert1968}, defines a
Steiner tree, as a tree that cannot be shortened by a small perturbation or by
``splitting'' a terminal and inserting a Steiner point.  Here a Steiner point is
an extra vertex that may be inserted into the graph and used as a ``connection
point'' for edges, as can be seen in \cref{fig:preliminaries-steiner-point}.

\begin{figure}[htbp]
\centering
\includegraphics[width=0.5\textwidth]{gfx/tikz/preliminaries-steiner-points}
\caption[Steiner point of isosceles triangle.]{The length of the tree on the left
  can be shortened by inserting an extra point as has been done on the
  right.\label{fig:preliminaries-steiner-point}}
\end{figure}

Another definition, used by Smith~\cite{Smith1992}, defines the Steiner tree a
tree with the following properties

\begin{enumerate}
\item It contains $N$ ``terminals'' $\vec{x}_1 \cdots \vec{x}_N$, and
  possibly $K$ additional ``Steiner points'' $\vec{x}_{N+1} \cdots
  \vec{x}_{N+K}$.
\item Each Steiner point has valence 3, and the edges emanating from it lie in a
  plane and have mutual angle $120^{\circ}$
\item Each terminal has valence between 1 and 3 (and generally $\le 2$).
\item $0 \le K \le N-2$.
\end{enumerate}

The two definitions given here are equivalent.  The second definition however explicitly
gives some rather useful information about the structure of the Steiner
tree.  This thesis therefore mainly uses the second definition, but might
use other equivalent definitions when convenient.

That the two definitions are indeed equivalent can seen by proving the
properties of the second definition.

The first property is by definition, and to name the terminals and the
Steiner points.

The second property can be shown in the following way.  First of consider that
any Steiner point must have at least valence three.  This should be obvious as
every point with a valence of $1$ simply can be removed to make the length the
same or shorter. This is true as these points do not lead to any terminals.
Every Steiner point with a valence of $2$ can be removed, and its edges be replaced
with one edge that is of the same length or shorter as per the triangle
inequality theorem.

We now show that no pair of edges emanating from a Steiner point can have a
mutual angle less than $120^{\circ}$.  Suppose the contrary is the true, and the
lines $PR$ and $RQ$ meet with angle $PRQ < 120^{\circ}$.  If we use the
mechanical interpretation given by Gilbert and Pollak~\cite{Gilbert1968} the two
edges ``pull'' on point $R$ with a resultant force of magnitude\NOTE{I don't
  quite get why this holds true?}
%
\begin{equation}
F = 2 \cos \theta / 2 > 1
\end{equation}
%
\TODO[inline]{Finish the part about the angles being no less than 120}

Using that the mutual angles may be no less than $120^{\circ}$ it follows that
every Steiner point can have a valence of at most $3$, as
$120^{\circ} = 360^{\circ}$.  We therefore have, using the first part about the
valence that a Steiner point always has a valence of $3$, and as the mutual angles
can be no less than $120^{\circ}$, that the angles between edges emanating from
a Steiner point must be exactly $120^{\circ}$.

\TODO[inline]{Write proof of third property}

The fourth property follows in the following way from property 3 and 4.  Any
tree has one edge fewer than vertices.  Thus a Steiner tree has $N+K-1$ edges.
Every edge has two endpoints meaning there are $2N+2K-2$ edges.  As every
Steiner point has a valence of $3$, they account for $3K$ of the endpoints.  For
the terminals we must split them in three groups -- $N_1$, those that have a
valence of $1$.  $N_2$, those that have a valence of $2$ and, $N_3$, those that
have a valence of $3$.  The terminals thus account for the rest of the
endpoints, written as $N_1 + 2 N_2 + 3 N_3 \ge N$.  We can then write the
equation as
%
\begin{align}
  3K + N_1 + 2N_2 + 3N_3 &= 2N + 2K - 2 \\
  K &= 2N - 2 - (N_1 + 2N_2 + 3N_3) \\
  0 \le K &\le N - 2
\end{align}

\subsection{Full Steiner tree}
\label{sec:full-steiner-tree}

A \gls{fst} is a Steiner tree where all terminals have valence 1, which is
equivalent to the tree having exactly $N-2$ Steiner points, the maximal number
of possible Steiner points.  This can easily be seen by considering the
definition of Steiner trees in \cref{sec:steiner-tree}.  This corresponds to the
proof of the fourth property, but here $N_1 = N$ and $N_2 = N_3 = 0$, meaning
that the number of Steiner points will be exactly $N-2$.

It can furthermore be shown that any Steiner tree is a union of edge-disjoint
\glspl{fst}, meaning that any edge of a \gls{smt} joining to terminals is
also an edge of \gls{mst} of the terminals.  This is relatively easily seen,
by realizing that every three points connected to a Steiner point is a \gls{fst}
for those three points.  These are then either connected to the rest of the tree
with a Steiner point, in which case they are part of a bigger \gls{fst}, or they
are connected to a terminal, in which case we can we can remove that edge
to have one \gls{fst} and the rest of the tree.

The topology of a \gls{fst} is called a full Steiner topology, also abbreviated
\gls{fst}.  Here we have the fixed coordinates of the terminals, but not the
coordinates of the Steiner points, and as such we only think about the
adjacencies of the points.  Whether we are talking about the topology or tree
will be clear from the context.

\subsection{Steiner minimal tree}
\label{sec:steiner-minimal-tree}

A minimal tree on $N$ points $\vec{x}_{1},\vec{x}_{2},\ldots,\vec{x}_{n}$ is the
tree having these points as its vertices and having the smallest possible sum of
the length of all its edges.  When we are allowed to insert any number of extra
points into the tree and connect the edges using these to further shorten the
tree this is known as a \gls{smt}~\cite{Gilbert1968}.

\NOTE[inline]{This could probably be fleshed out a bit more. Write about SMTs
  being steiner trees or vice versa (See Smith).}

\section{Euclidean Steiner tree problem}
\label{sec:eucl-stein-tree}

The \gls{estp} is the problem of finding the \gls{smt} for the set of points $N$
in Euclidean $d$-space.  I.e.\ each point $x_{i}, 1 \le i \le n$ is a
$d$-vector, and the metric used to calculate the length of the edges connecting
the points is the $L^2$-norm, the Euclidean norm.

The work of this thesis is only concerned with the Euclidean version of the
Steiner tree problem in $d$-space.

\chapterbreak{}

%%% Local Variables:
%%% mode: latex
%%% TeX-master: "../../main"
%%% End:
